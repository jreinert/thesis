\subsection{API-First}
\label{ssec:ea-api-first}

Wie IOOI, soll auch das neue System nach dem \index{API-First} \cite{api-first}
Designprinzip gestaltet werden (Abbildung \ref{fig:api-first-architektur}).  Im
Gegensatz zur in Abbildung \ref{fig:klassische-rails-architektur} dargestellten
klassischen Rails Architektur wird nur eine Serverseitige Applikation benötigt.
Zusätzlich kann ein einfacher Fileserver zum ausliefern von nicht
serialisierbaren Daten (Assets) wie Bildern oder Videos eingesetzt werden.

\begin{figure}[h]
	\centering
	\includestandalone[width=\textwidth]{fig_api_first_architektur}
	\caption{API-First Architektur}
	\label{fig:api-first-architektur}
\end{figure}

\begin{figure}[h]
	\centering
	\includestandalone[width=\textwidth]{fig_klassische_rails_architektur}
	\caption{Klassische Rails MVC-Architektur}
	\label{fig:klassische-rails-architektur}
\end{figure}

Im Folgenden wird auf die Vorteile von API eingegangen. 

\subsubsection{DRY}
\label{sssec:eaa-dry}

In API-First Architekturen ist es Einfacher das \index{DRY}-Prinzip (Don't
repeat yourself) einzuhalten. Wie in Abbildung \ref{fig:api-first-architektur}
zu erkennen, muss nur noch in der API Domain-Logik implementiert werden.  CMS,
Webseite und Mobile Applikation greifen alle nur noch auf die API zu die
CRUD-Operationen auf den Daten abstrahiert.

\subsubsection{Client-Unabhängigkeit}
\label{sssec:eaa-client-unabhaengigkeit}


