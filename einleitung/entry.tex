\chapter{Einleitung}
\label{chap:einleitung}

Diese Bachelor-Thesis mit dem Titel \emph{Collaborative Content Creation -- Ein
System zum konfliktfreien Erzeugen und Pflegen von Inhalten mit asynchroner
Cache-Verwaltung und JSON-API} wurde zur Erlangung des akademischen Grades
Bachelor of Science (B.Sc.) an der \emph{Hochschule RheinMain} im Studiengang
\emph{Angewandte Informatik} des Fachbereichs \emph{Design Informatik Medien}
in Wiesbaden von Joakim Reinert (\mailto{studium@jreinert.com}) verfasst.  Die
Arbeit wurde im Wintersemester 2015/16 fertiggestellt.

Die Thesis befasst sich mit neuen Web-Technologien und
Performance-Optimierungen in Form von Cache-Verwaltung im Web-Kontext.  Der
Fokus liegt hierbei auf der Konzeption und Implementierung eines
Content-Management-Systems bestehend aus einem in ECMA-Script-6 entwickelten
Frontend und einer losgelösten, in Crystal entwickelten JSON-API mit
eingebautem Serialisierungscache.

Die Arbeit wurde extern in Zusammenarbeit mit der Firma \emph{\mesods} in
Frankfurt am Main bearbeitet.  Wenn Teile \emph{nicht} vom Autor selbst
verfasst sind, sind diese entsprechend gekennzeichnet und deren Urheber
genannt.

Der Betreuer der Thesis im Unternehmen war der Mitgründer und Geschäftsführer
\mbox{Mathias Wollin} (\mailto{wollin@meso.net}).  Im folgenden soll die Firma
kurz vorgestellt werden.

\subimport{meso/}{entry}
\subimport{problemstellung/}{entry}
\subimport{anforderungen/}{entry}
\subimport{loesungsansaetze/}{entry}
