\subsection{\mesods}
\label{ssec:em_meso_digital_services}

\mesods~-- im Folgenden mit \emph{MESO DS} abgekürzt -- entwickelt
Webanwendungen -- vor allem Content-Management-Systeme -- sowie mobile
Applikationen und andere auf Webtechnologien basierende Dienste.  Zur Zeit des
Verfassens dieser Thesis kommt bei vielen Projekten das Web-Framework
Ruby-On-Rails zum Einsatz, doch auch Sinatra~\cite{sinatra},
Backbone~\cite{backbone}, Ember~\cite{ember}, Angular~\cite{angular} und
Meteor~\cite{meteor} sind als Technologien vertreten.  Der Fokus verschiebt
sich aber immer stärker in Richtung Reaktivität bietender Frameworks oder
--Technologien wie Meteor.

In gemeinsamen Projekten mit MESO DI geht es oft um Messe-Exponate, die von MESO
DI entwickelt werden und parallel von MESO DS mit einer begleitenden
Webapplikation oder mobilen Anwendung versehen werden.
