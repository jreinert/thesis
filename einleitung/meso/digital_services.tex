\subsection{\mesods}
\label{ssec:em-meso-digital-services}

\mesods~-- im Folgenden mit \emph{MESO DS} abgekürzt -- entwickelt
Webanwendungen -- vor allem Content-Management-Systeme -- sowie mobile
Applikationen und andere auf Webtechnologien basierende Dienste.  Zur Zeit des
Verfassens dieser Thesis kommt bei vielen Projekten das Web-Framework
Ruby-On-Rails zum Einsatz, doch auch Sinatra~\cite{sinatra},
Backbone~\cite{backbone}, Ember~\cite{ember}, Angular~\cite{angular} und
Meteor~\cite{meteor} sind als Technologien vertreten.  Der Fokus verschiebt
sich aber immer stärker in Richtung Reaktivität bietender Frameworks oder
--Technologien wie Meteor.

In gemeinsamen Projekten mit MESO DI wird oft für Messen, auf
denen ein Kunde ein Exponat ausstellt, welches von MESO DI entwickelt wird,
parallel eine Webapplikation oder mobile Anwendung, die auf den selben
Datenstamm zurückgreift, von MESO DS ausgearbeitet.
