\section{\meso}
\label{sec:e_meso}

\meso~wurde 1997 von Stefan Ammon, Michael Höpfel, Karl Kliem, Sebastian
Oschatz und Max Wolf gegründet. Ein Jahr später gründeten Mathias Wollin,
Martin Schuster und Markus Schnabel \emph{aspekt1 --- significant media works}.

Im Jahre 2004 bildeten \meso~und Teile von aspekt1 drei eigenständige
Firmenteile unter dem Namen MESO: MESO Web Scapes (Stefan Ammon, Mathias
Wollin, und Martin Schuster), MESO Digital Interiors (Sebastian Oschatz und Max
Wolf) und MESO Image Spaces (Michael Höpfel). Karl Kliem gründete im selben
Jahr seine eigene Firma \emph{Dienststelle} in Frankfurt am Main.

Aus MESO Web Scapes entstand 2015 die \mesods. Gründer sind  Mathias Wollin und
Martin Schuster.  Stefan Ammon baute sein eigenes Unternehmen \emph{Neue
Räumlichkeit} auf.  Ende 2015 verstarb Martin Schuster plötzlich, sodass die
Unternehmensführung nun vollständig von Mathias Wollin übernommen wurde.

\meso~ist also heute ein Firmenkollektiv bestehend aus drei Unternehmen:
\mesodi, \mesods~und \mesois.  Die Gründer und Geschäftsführer der drei Firmen
kommen zum größten Teil aus der Design-Branche; viele von ihnen sind Absolventen
der \emph{\href{http://hfg-offenbach.de}{Hochschule für Gestaltung Offenbach am
Main}}.  Dies macht \meso~zu einem interessanten Arbeitgeber, der sich von
vielen klassischen Informatik-Unternehmen unterscheidet, da hier viel Wert auf
Ästhetik gelegt und gerne mit Neuem experimentiert wird.  Diese Einstellung
zieht sich oft in Form von Anspruch an Code-Qualität oder dem agilen
Produktiveinsatz von Software bis auf die Programmierebene durch.

Besonders zwischen \mesodi~und \mesods~besteht reger Austausch und
Zusammenarbeit in Form von gemeinsamen Projekten.  Die Aufgabengebiete dieser
Unternehmen sind nachfolgend erläutert.

\subsection{\mesodi}
\label{sec:em-meso-digital-initeriors}

\mesodi~-- im folgenden mit \emph{MESO DI} abgekürzt -- konzipiert, gestaltet
und entwickelt digitale Installationen für Messen, Museen und Events.  Dabei
arbeiten sie weitestgehend mit dem von \meso~selbst entwickelten und von
\mesois~vertriebenen Toolkit VVVV~\cite{v4}.  VVVV nutzt DirectX und ermöglicht
Video-Rendering in Echtzeit.  Zum Programmieren der Installationen (sogenannten
\enquote{Patches}) wird ein knotenbasiertes visuelles Interface genutzt.

\subsection{\mesods}
\label{ssec:em_meso_digital_services}

\mesods~-- im Folgenden mit \emph{MESO DS} abgekürzt -- entwickelt
Webanwendungen -- vor allem Content-Management-Systeme -- sowie mobile
Applikationen und andere auf Webtechnologien basierende Dienste.  Zur Zeit des
Verfassens dieser Thesis kommt bei vielen Projekten das Web-Framework
Ruby-On-Rails zum Einsatz, doch auch Sinatra~\cite{sinatra},
Backbone~\cite{backbone}, Ember~\cite{ember}, Angular~\cite{angular} und
Meteor~\cite{meteor} sind als Technologien vertreten.  Der Fokus verschiebt
sich aber immer stärker in Richtung Reaktivität bietender Frameworks oder
--Technologien wie Meteor.

In gemeinsamen Projekten mit MESO DI geht es oft um Messe-Exponate, die von MESO
DI entwickelt werden und parallel von MESO DS mit einer begleitenden
Webapplikation oder mobilen Anwendung versehen werden.

