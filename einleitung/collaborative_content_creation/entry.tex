\section{Collaborative Content Creation}
\label{sec:e_collaborative_content_creation}

Die Software, die im Rahmen dieser Thesis entwickelt wurde, trägt den Namen
Collaborative Content Creation oder kurz: CCC.  Hierbei handelt es sich genauer
um mehrere Software-Projekte, die speziell für Teilaufgaben des Gesamtproduktes
CCC entwickelt wurden, zum Teil aber breitere Verwendung finden und deshalb
quelloffen unter der MIT-Lizenz \cite{mit} veröffentlicht wurden.  Alle
quelloffenen Teilprojekte sind auf \href{https://github.com}{Github} unter
\texttt{\href{https://github.com/jreinert}{https://github.com/jreinert}} zu
finden und zusammen mit den nicht-quelloffenen Teilen auf der beiliegenden
Daten-CD enthalten.

Auf die genaue Problemstellung oder Motivation, die Anforderungen und
entwickelten Lösungsansätze wird in \cref{chap:konzeption} eingegangen.  Um
dem/der Leser/-in ein Grundwissen für die verwendeten Programmiertechniken,
Architekturen und Technologien zu vermitteln, werden diese in
\cref{chap:grundlagen} behandelt.  Die Teilprojekte und deren Aufgaben werden in
den Kapiteln \ref{chap:implementierung_backend} und
\ref{chap:implementierung_frontend} vorgestellt.  In \cref{chap:testing} wird
erwiesen, wie die entwickelte Software auf ihre Korrektheit geprüft wurde.
Schließlich wird in \cref{chap:diskussion} über aufgetretene Probleme diskutiert
und die Arbeit reflektiert.  Am Ende dieser Ausarbeitung befindet sich
zusätzlich zum Quellenverzeichnis ein Stichwortverzeichnis zum Nachschlagen von
verwendeten Fachbegriffen.  Aus dem Anhang sind längere Tabellen oder
Zusatzinformationen zu entnehmen, worauf in den entsprechenden Fällen
hingewiesen wird.
