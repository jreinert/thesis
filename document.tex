\frontmatter
\maketitle

\chapter{Abstract}
\label{abstract}

Diese Bachelor-Thesis umfasst die Konzeption und Implementierung eines
skalierbaren Content-Management-Systems, das über ein reaktives
Benutzerinterface verfügt sowie eine API bereitstellt, die einen hierarchischen
Cache über die Inhalte verwaltet.  Das System ermöglicht das kollaborative
Arbeiten an hierarchischen Inhalten, indem Änderungen in Echtzeit zwischen den
verbundenen Clients synchronisiert werden.  Es werden verschiedene
Lösungsansätze basierend auf unterschiedlichen Webtechnologien gegeneinander
abgewägt und ein modulares, auf Crystal und Redux basierendes Konzept
sowie die daraus resultierende Implementierung vorgestellt.

\tableofcontents

\mainmatter
\import{einleitung/}{entry}
\import{konzeption/}{entry}
\import{grundlagen/}{entry}
\import{implementierung_backend/}{entry}
\import{implementierung_frontend/}{entry}
\import{testing/}{entry}
\import{diskussion/}{entry}

\backmatter
\printbibliography%
\printindex
\import{anhang/}{entry}
