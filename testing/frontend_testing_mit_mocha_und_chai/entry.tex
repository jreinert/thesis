\section{Frontend-Testing mit Mocha und Chai}
\label{sec:t_frontend_testing_mit_mocha_und_chai'}

Das Frontend-Testing wurde mit dem Test-Framework \emph{Mocha}~\cite{mocha}
durchgeführt.  Mocha unterstützt unter anderem asynchrone Testfälle und
Diff-Ansichten zwischen erwarteten und eigentlichen Ergebnissen.
Die Diff-Ansichten sind gerade beim Testen von Actions hilfreich, da schnell zu
erkennen ist, welche Felder der erwarteten und eigentlichen Actions voneinander
abweichen.

Mocha bietet von sich aus keine DSL für Assertions.  Dadurch ist dem Entwickler
frei überlassen welche Assertion-Bibliothek genutzt werden soll.  Für diese
Arbeit wurde \emph{Chai} mit BDD-Assertions\footnote{Behavior-Driven
Development} genutzt.

Als eigentlicher Test-Runner wurde \emph{Karma}~\cite{karma} verwendet.  Karma
kann von der Kommandozeile gestartet und in Webpack so eingebunden werden, dass
alle Tests bei jeder Dateiänderung automatisch neu ausgeführt werden.

\subsection{Action-Tests}
\label{sec:tf_action_tests}

In Action-Tests werden Action-Creators getestet.  Es wird geprüft, ob sie für
die Eingabeparameter korrekte Action-Objekte zurückgeben.  Bei Actions, die
API-Requests ausführen, wird geprüft, ob diese ebenfalls korrekt sind und die
Responses der API richtig bearbeitet werden.

Um die Requests und Responses zur und von der API kontrollieren zu können, wurde
das NPM-Paket \emph{fetch-mock}~\cite{fetch-mock} genutzt.  Mit dieser
Bibliothek wird die globale Funktion \code{fetch(...)} überschrieben und das
Überwachen und Überprüfen von Requests ermöglicht.  Zusätzlich können zu den
Requests beliebige Responses definiert werden.

Zum Überprüfen der Action-Dispatchtes auf dem Store wurde eine Mock-Klasse
geschrieben, die einen Store simuliert und Einblick in die ausgelösten Actions
erlaubt.

\cref{lst:action_tests} zeigt die Nutzung des Store-Mocks.  Der hier getestete
Action-Creator \code{add} gibt ein Promise-Objekt zurück.  Es wird genutzt, um
einen asynchronen Test durchführen zu können.  In der \code{then}-Funktion
wird der zum Store-Mock gehörende \code{actionSpy} genutzt, um zu prüfen,
ob die ausgelösten Actions korrekt sind.

\lstinputlisting[%
	float,
	language=ES6,
	caption={Frontend Testing: Action Tests mit Store-Mock},
	label={lst:action_tests}
]{action_tests.js}

In \cref{lst:action_tests_api} werden die vom selben Action-Creator ausgeführten
API-Requests überprüft.

\lstinputlisting[%
	float,
	language=ES6,
	caption={Frontend Testing: Action Tests mit fetch-mock},
	label={lst:action_tests_api}
]{action_tests_api.js}

\subsection{Reducer-Tests}
\label{sec:tf_reducer_tests}

Reducer-Tests sind einfacher umzusetzen als Action-Tests, da es sich hier um
simple Funktionen ohne Seiteneffekte handelt.  Den Reducern wird lediglich ein
State und eine Action übergeben und anschließend der zurückgegebene neue State
überprüft.

\cref{lst:selected_item_reducer} zeigt den Test vom \code{selectedItem}-Reducer
mit der Action \code{CHANGE\_ITEM}.

\lstinputlisting[%
	float,
	language=ES6,
	caption={Reducer-Tests},
	label={lst:selected_item_reducer}
]{reducer.js}

