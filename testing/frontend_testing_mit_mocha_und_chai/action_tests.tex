\subsection{Action-Tests}
\label{sec:tf_action_tests}

In Action-Tests werden Action-Creators getestet.  Es wird geprüft, ob sie für
die Eingabeparameter korrekte Action-Objekte zurückgeben.  Bei Actions, die
API-Requests ausführen, wird geprüft, ob diese ebenfalls korrekt sind und die
Responses der API richtig bearbeitet werden.

Um die Requests und Responses zur und von der API kontrollieren zu können, wurde
das NPM-Paket \emph{fetch-mock}~\cite{fetch-mock} genutzt.  Mit dieser
Bibliothek wird die globale Funktion \code{fetch(...)} überschrieben und das
Überwachen und Überprüfen von Requests ermöglicht.  Zusätzlich können zu den
Requests beliebige Responses definiert werden.

Zum Überprüfen der Action-Dispatchtes auf dem Store wurde eine Mock-Klasse
geschrieben, die einen Store simuliert und Einblick in die ausgelösten Actions
erlaubt.

\cref{lst:action_tests} zeigt die Nutzung des Store-Mocks.  Der hier getestete
Action-Creator \code{add} gibt ein Promise-Objekt zurück.  Es wird genutzt, um
einen asynchronen Test durchführen zu können.  In der \code{then}-Funktion
wird der zum Store-Mock gehörende \code{actionSpy} genutzt, um zu prüfen,
ob die ausgelösten Actions korrekt sind.

\lstinputlisting[%
	float,
	language=ES6,
	caption={Frontend Testing: Action Tests mit Store-Mock},
	label={lst:action_tests}
]{action_tests.js}

In \cref{lst:action_tests_api} werden die vom selben Action-Creator ausgeführten
API-Requests überprüft.

\lstinputlisting[%
	float,
	language=ES6,
	caption={Frontend Testing: Action Tests mit fetch-mock},
	label={lst:action_tests_api}
]{action_tests_api.js}
