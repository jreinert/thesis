\subsection{Reducer-Tests}
\label{sec:tf_reducer_tests}

Reducer-Tests sind einfacher umzusetzen als Action-Tests, da es sich hier um
simple Funktionen ohne Seiteneffekte handelt.  Den Reducern wird lediglich ein
State und eine Action übergeben und anschließend der zurückgegebene neue State
auf Korrektheit geprüft.

\cref{lst:selected_item_reducer} zeigt den Test vom \code{selectedItem}-Reducer
mit der Action \code{CHANGE\_ITEM}.

\lstinputlisting[%
	float,
	language=ES6,
	caption={Reducer-Tests},
	label={lst:selected_item_reducer}
]{reducer.js}
