\subsection{BSON-Mapping}
\label{ssec:bm_bson_mapping}

Um für beliebige Klassen eine einfache Syntax zum Bestimmen einer Abbildung
nach BSON bereitzustellen, wurde das Macro \code{mapping} in der Klasse
\code{BSON} implementiert.  Dem Macro wird ein Hash-Literal übergeben, das
Attribute eines Objekts Werten eines BSON-Dokuments zuordnet.  Hier können
auch Optionen wie Default-Werte übergeben werden.  \cref{lst:bson_mapping} zeigt
ein Beispiel zur Nutzung des Macros.

\lstinputlisting[%
	float,%
	label={lst:bson_mapping},%
	caption={MongoDB-Treiber: BSON-Mapping},%
	language=Crystal%
]{mapping.cr}

In Zeile 30 und 31 werden einfache Zuordnungen von String-Werten definiert.  In
den Zuordnungen in Zeile 28 und 31 bis 35 werden die Optionen \code{key} und
\code{converter} genutzt.  Mit \code{key} kann ein von dem Namen der
Instanzvariable abweichender Schlüssel im BSON-Dokument genutzt werden.  Mit
\code{converter} kann ein Objekt angegeben werden, das für die Serialisierung
des Wertes zuständig ist.  Auf dem Objekt müssen die Methoden
\code{from\_bson(bson\_value)} und \code{to\_bson(value, appender)} aufgerufen
werden können.  So lassen sich komplexere BSON-Serialisierungen implementieren.

Das Macro implementiert die Methoden \code{to\_bson()} und
\code{to\_bson(appender)} sowie einen Konstruktor \code{initialize(bson)}.  Der
Konstruktor übernimmt die Werte aus dem übergebenen BSON-Dokument für die
gleichnamigen Instanzvariablen.  Falls Default-Werte angegeben wurden, werden
diese bei fehlenden Feldern im BSON übernommen.  Durch die
\code{to\_bson()}-Methode wird ein neues BSON-Dokument erzeugt.  Dabei wird es
durch Aufrufe der \code{to\_bson(appender)}-Methoden der den Instanzvariablen
zugewiesenen Objekte befüllt.  Alternativ kann dies durch definierte Konverter
geschehen.  Die Methode \code{to\_bson(appender)} ermöglicht das Verschachteln
von Mappings, da beim Serialisieren und Deserialisieren von Typen die Methode
\code{to\_bson(appender)} sowie der Konstruktor \code{initialize(bson)} genutzt
werden.  Die Methode und der Konstruktor wurden für viele Crystal-Typen wie zum
Beispiel \code{String}, \code{Integer}, \code{Array} oder \code{Object}
implementiert.
