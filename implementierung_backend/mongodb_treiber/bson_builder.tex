\subsection{BSON-Builder}
\label{ssec:bm_bson_builder}

Durch die Arbeit an der in \cref{sec:b_jsonapi_serializer} vorgestellten
Bibliothek \emph{JSONApi} konnte die daraus gewonnene Erkenntnis bezüglich der
Crystal-internen Mechanismen zur JSON-Serialisierung auch für BSON genutzt
werden.

Die Serialisierung zu BSON unterscheidet sich von JSON in der Hinsicht, dass
BSON-serialisierte Daten immer in Dokumenten zusammengefasst sind, was in JSON
Objekten entsprechen würde.  So kann beispielsweise ein String oder eine Zahl
nur als Wert zu einem Schlüssel gespeichert werden.

Zum Erzeugen von BSON-Dokumenten wurden die Klassen \code{BSON::Builder} und
\code{BSON::ArrayBuilder} sowie die statische Methode \code{build(\&block)} zur
Klasse \code{BSON} implementiert.  Die Methode übergibt dem Code-Block eine
Instanz von \code{BSON::Builder} und hat als Rückgabewert das resultierende
BSON-Dokument.

Zur eigentlichen Serialisierung, das heißt dem Speichern von diversen
Crystal-Typen in BSON-Dokumenten, wurde eine Klasse \code{BSON::Appender}
implementiert.  Die Klasse besitzt die Methoden \code{<<(value)},
\code{document(\&block)} sowie \code{array(\&block)}.  Der Konstruktor der
Klasse erhält das BSON-Dokument, dem ein Wert hinzugefügt werden soll, sowie
den Schlüssel zum Wert.  Mit den Methoden \code{document(\&block)} und
\code{array(\&block)} können sogenannte \stichwort[BSON]{embedded
Documents}~\cite{mongo-embedded-docs} erzeugt werden, was einer Verschachtelung
von Objekten beziehungsweise Arrays in JSON entspräche.  Den Code-Blöcken werden
Instanzen von \code{BSON::Builder} beziehungsweise \code{BSON::ArrayBuilder}
übergeben.

Die Klasse \code{BSON::Builder} besitzt die Methoden \code{field(key, \&block)}
sowie \code{field(key, value)}.  Die erstere übergibt dem Code-Block eine
\code{BSON::Appender}-Instanz, die genutzt werden kann, um ein Objekt zu
serialisieren.  Die Methode \code{field(key, value)} ruft
\code{field(key, \&block)} auf.  Im Code-Block wird auf \code{value} die Methode
\code{to\_bson(appender)} mit dem übergebenen \code{BSON::Appender} aufgerufen.

Die in \cref{fig:mongocr_builder_klassendiagramm} zusammengefasste Architektur
ermöglicht es ähnlich zur JSON-Implementierung von Crystal die Komplexität
einer Serialisierung zu verteilen und dadurch wartbaren und testbaren Code zu
schreiben.

\begin{figure}
	\centering
	\includestandalone[width=\textwidth]{fig_builder_klassendiagramm}
	\caption{MongoDB-Treiber: Klassendiagramm zu BSON-Builder}
	\label{fig:mongocr_builder_klassendiagramm}
\end{figure}
