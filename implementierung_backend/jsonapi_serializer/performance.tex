\subsection{Performance}
\label{ssec:bj_performance}

Zur Performance-Messung der Serialisierung wurde ein Iterator implementiert,
der zufällige Personen-Resourcen nach dem Schema in
\cref{lst:json_api_resource_macros} erzeugt.  Es wurden Requests mit
unterschiedliche Anzahlen von Resources simuliert, um die Auswirkungen des
Cachings bei steigender Größe der serialisierten Daten zu beobachten.  In
\cref{fig:json_api_performance} sind die Messdaten in einem Diagramm
zusammengefasst.

\begin{figure}
	\center
	\includestandalone[width=\textwidth]{fig_performance}
	\caption{JSON-API: Performance}
	\label{fig:json_api_performance}
\end{figure}

Wie zu erwarten, wird bei größeren Request-Payloads ein proportional größerer
Cache benötigt, da ab einer Payload-Größe von durchschnittlich ungefähr einem
Hundertstel des Caches eine Sättigung erreicht wird und die zum Einbruch der
Response-Geschwindigkeit führt.  Ab dieser Sättigung nimmt das Bereinigen des
Caches von alten Einträgen den Großteil der Rechenleistung in Anspruch.

Aus den Messungen ergibt sich ebenfalls, dass der Cache für um eine
Zehnerpotenz schnellere Request-Zeiten sorgt im Vergleich zu einer
Implementierung ohne Cache.
