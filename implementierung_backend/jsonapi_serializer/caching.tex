\subsection{Caching}
\label{ssec:bj_caching}

Das Caching-System von JSONApi nutzt die in
\cref{ssec:bj_json_serialisierung_in_crystal} erläuterten Vorteile der
JSON-Serialisierung in Crystal aus.  Dazu wurde ein Modul \code{Cacheable}
implementiert.  Wie aus \cref{fig:json_api_klassendiagramm} ausgeht,
ist in \code{Cacheable} die Methode \code{to\_json(io)} definiert.  Die Methode
ruft wiederum \code{to\_cached\_json(io)} auf, die von allen Klassen, die
\code{Cacheable} einbinden, implementiert werden muss.

Die Methode \code{to\_json} ruft \code{fetch} mit dem von \code{cache\_key}
ausgegebenen Schlüssel und dem übergebenen \code{IO}-Objekt auf.  Zusätzlich
wird ein Code-Block übergeben, der im Falle eines Cache-Misses den JSON-String
in ein dem Block übergebenes \code{IO}-Objekt schreiben soll.  In
\cref{fig:json_api_caching_sequenzdiagramm} ist der Ablauf eines
\code{fetch}-Aufrufs dargestellt.

Die \code{fetch}-Methode in \code{Cache} prüft, ob bereits ein
Cache-Record für den übergebenen Schlüssel existiert.  Wenn dies der
Fall ist, wird der Inhalt des Records in das übergebene \code{IO}-Objekt
kopiert und dabei der \code{last\_accessed}-Wert des Records mit der aktuellen
Uhrzeit überschrieben.  Anschließend wird der Cache-Hit-Zähler erhöht.

Wird zu dem Schlüssel kein Record gefunden, wird der Cache-Miss-Zähler erhöht
und ein neuer Record initialisiert.  Dem Konstruktor des neuen Records wird das
\code{IO}-Objekt sowie der übergebene Code-Block weitergereicht.  Der
Konstruktor ruft nun den Code-Block auf und übergibt diesem die Record-Instanz.
Die \code{write}-Methode von \code{Cache::Record} ist so überschrieben, dass
sie die Bytes zusätzlich in das dem Konstruktor übergebene \code{IO}-Objekt
schreibt.  Anschließend wird mit der \code{cleanup}-Methode eine Art
Garbage-Collection auf dem Cache ausgeführt.  Die Methode löscht so lange alte
Cache-Records, bis die Maximale Cache-Größe nicht mehr überschritten ist.

\begin{figure}
	\center
	\includestandalone[width=\textwidth]{fig_caching_sequenz_diagramm}
	\caption{JSON-API: Caching Sequenzdiagramm}
	\label{fig:json_api_caching_sequenzdiagramm}
\end{figure}
