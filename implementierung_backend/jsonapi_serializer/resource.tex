\subsection{Resource}
\label{ssec:bj_resource}

Wie in \cref{sec:gj_resources} erläutert, repräsentieren Resources in der
JSON-API-Spezifikation die Entitäten eines Datenbankschemas.  In der
Crystal-Bibliothek wurde, wie bereits in \cref{fig:json_api_klassendiagramm}
ersichtlich, eine abstrakte Klasse \code{Resource} angelegt.  Zum Definieren
von Attributen und Relationen wurden die Macros \code{attributes(map)} und
\code{relationships(map)} angelegt, die jeweils die Methoden
\code{get\_attributes()} und \code{attributes(object, io)} beziehungsweise
\code{get\_relationships()} und \code{relationships(object, io)}
implementieren.  Die Methoden \code{get\_attributes()} und
\code{get\_relationships()} geben eine Hashmap der Attribut-/Relationsnamen mit
den aktuellen Werten zurück.  Die Methoden \code{attributes(object, io)} und
\code{relationships(object, io)} hingegen dienen zur Serialisierung der
jeweiligen Daten und werden von \code{to\_cached\_json(io)} aufgerufen.

Den Macros werden die Attribute beziehungsweise Relationen und deren
Eigenschaften wie Typ oder Fremdschlüssel in einem Hash-Literal übergeben.
Eine mögliche Anwendung der Macros ist aus \cref{lst:json_api_resource_macros}
zu entnehmen.  Aus dem Beispiel wird ersichtlich, dass das gesamte Schema einer
Applikation durch das Implementieren von Unterklassen von \code{Resource}
abgebildet werden kann.

\lstinputlisting[%
	language=Crystal,%
	numbers=left,%
	label={lst:json_api_resource_macros},%
	caption={JSON-API: Resource Macros},%
	float%
]{person.cr}
