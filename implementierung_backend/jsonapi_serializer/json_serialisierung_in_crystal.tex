\subsection{JSON-Serialisierung in Crystal}
\label{ssec:bj_json_serialisierung_in_crystal}

Zunächst soll erläutert werden, wie die JSON-Serialisierung in Crystal
implementiert wird.  Sobald die eingebaute JSON-Bibliothek mit \code{require
"json"} eingebunden wird, definiert sie die Methoden \code{json\_object} sowie
\code{json\_array} für das \code{IO}-Modul.  Dieses Modul wird von allen
IO-Klassen wie Sockets, oder Files eingebunden.  Zusätzlich implementiert es
die Methode \code{to\_json(io)} auf einigen Standard-Klassen sowie
\code{String}, \code{Integer}, \code{Float}, \code{Hash}, \code{Array},
\code{Nil} und \code{Boolean}.  Die Klasse \code{Object}, von der alle weiteren
Klassen erben, implementiert die Methode \code{to\_json}\footnote{diesmal ohne
Parameter}, die einen \code{String::Builder} erzeugt und diesen an
\code{to\_json(io)} übergibt.

Um ein beliebiges Objekt in einen JSON-String serialisieren zu können, muss
dessen Klasse die Methode \code{to\_json(io)} implementieren.  Ein Vorteil
dieser Art der Implementierung ist, dass sich \code{to\_json(io)}-Aufrufe
beliebig verschachteln lassen, indem das übergebene \code{IO}-Objekt
weitergereicht wird.  Desweiteren ist möglich, JSON zu streamen.  Es muss also
nicht der gesamte resultierende JSON-String im Speicher gehalten werden, um ihn
später über ein \code{IO}-Objekt in ein File oder Socket zu schreiben.
