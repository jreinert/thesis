\subsection{Middleware}
\label{ssec:ba_middleware}

Für die API wurden einige sogenannte \stichwort{Middlewares} entwickelt, um
Aktionen auszuführen, die nicht Routen-basiertem Request-Handling zuzuordnen
sind.  Bei den Middlewares handelt es sich um Unterklassen von HTTP::Handler.

\subsubsection{Authentifizierung}
\label{sssec:bam_authentifizierung}

Die Authentifizierung von CCC beruht auf Token- und Passwort-basierende
Verfahren.  Hierzu wird der Request-Header \code{Authorization} genutzt.  Die
Middleware \code{Auth} validiert keine Authentifizierung, sondern stellt
lediglich ein Service-Object \code{Authenticator} bereit, die im Controller
genutzt werden kann um Requests zu authentifizieren.

\subsubsection{Cross-Origin-Resource-Sharing}
\label{sssec:bam_cors}

Da es sich beim CCC-Frontend um eine pure JavaScript-Applikation handelt, die
ihre Daten über die API bezieht, müssen für alle Requests
Cross-Origin-Resource-Sharing- oder CORS-Header gesetzt werden. Über
CORS-Header kann ein Web-Browser prüfen, ob die Applikation befugt ist den
Inhalt zu beziehen.  Damit kann sichergestellt werden, dass der JavaScript-Code
vertrauenswürdig ist.  Diese Header werden von der Middleware \code{Cors}
gesetzt.

Über den Header \code{Access-Control-Allow-Origin} teilt der API-Server dem
Browser mit, welche/-r Host(s) vertrauenswürdig ist/sind.  Der Host, über den
das JavaScript geladen wurde, muss also über den Header freigegeben worden
sein.  Die erlaubten Hosts werden dem Konstruktor der Middleware übergeben.
Weitere von der gesetzte Header sind \code{Access-Control-Allow-Methods} und
\code{Access-Control-Allow-Headers}.  Diese dienen zum Einschränken von
erlaubten HTTP-Verben und -Headern.  Es wurden alle HTTP-Verben und die Header
\code{Authorization} und \code{Accept} erlaubt.

\subsubsection{MongoDB}
\label{sssec:bam_mongodb}

Die Middleware \code{Mongo} stellt den Controllern eine Verbindung zur
Datenbank zur Verfügung.  Hierzu wird dem Konstruktor die URL zu MongoDB und
der Datenbankname übergeben.

\subsubsection{Statische Dateien}
\label{sssec:bam_statische_dateien}

Um Statische Dateien wie Bilder, CSS oder JavaScript für das Frontend
auszuliefern wurde eine Middleware \code{StaticFiles} implementiert.  Dieser
wird ein Pfad im Konstruktor übergeben, der die Statischen Dateien enthält.
Wird eine URL aufgerufen, die eine Datei innerhalb des Pfades referenziert,
prüft die Middleware den MIME-Type der Datei, setzt den Header
\code{Content-Type} und schreibt den Inhalt der Datei in den Response-Body.

\subsubsection{Broadcast}
\label{sssec:bam_broadcast}

Um Client-zu-Client-Kommunikation zu ermöglichen wurde die Middleware
\code{Broadcast} umgesetzt.  Sie akzeptiert WebSocket-Verbindungen von Clients.
Beim Öffnen einer Verbindung wird eine Authentifizierung des Clients über einen
Token erwartet.  Sobald die Authentifizierung erfolgt, können über die
Verbindung Broadcast-Nachrichten gesendet und empfangen werden.
