\section{CCCApi}
\label{sec:b_cccapi}

Das eigentliche Backend von CCC wurde im Rahmen der Crystal-Applikation
\emph{CCCApi} implementiert.  Hier wurden die in
\cref{sec:b_jsonapi_serializer}, \ref{sec:b_crouter} und
\ref{sec:b_mongodb_treiber} beschriebenen Bibliotheken zu einem ausführbaren
Programm zusammengeführt.

CCCApi nutzt Crouter zum Definieren von RESTful Routes und JSONApi sowie den
MongoDB Treiber und die dafür entwickelten Methoden zur Serialisierung der
Daten zu JSON beziehungsweise BSON.

Die Domain-Logik wurde aufgeteilt in Repositories nach dem
Repository-Entwurfsmuster~\cite{repository-pattern}, Controller und sogenannte
\stichwort{Service-Objects}~\cite{service-objects}.  Auf die einzelnen
Komponenten wird im Folgenden näher eingegangen.

\subsection{Middleware}
\label{ssec:ba-middleware}

\subsection{Abstrakte Repository-Klasse}
\label{ssec:ba_abstrakte_repository_klasse}

Wie bereits erwähnt, wurde in CCCApi vom Repository-Entwurfsmuster Gebrauch
gemacht.  Dazu wurde eine abstrakte Klasse \code{Repository} konzipiert, über
die Datenbankabfragen und -befehle abstrahiert sind.  Hierbei handelt es sich
um eine Generische Klasse.  Der generische Typ \code{T} ist wird mit einer
Unterklasse von \code{JSONApi::Resource} belegt.  In
\cref{fig:cccapi_repository} sind die Methoden der Klasse zusammengefasst.

\begin{figure}
	\centering
	\includestandalone{fig_repository_klassendiagramm}
	\caption{CCCApi: Repository}
	\label{fig:cccapi_repository}
\end{figure}

Die Macro-Methode \code{collection\_name()} hat den Klassennamen als
Rückgabewert.  Dieser wird genutzt um im Konstruktor die Instanzvariable
\code{@collection} zu initialisieren.  Alle anderen Methoden arbeiten auf
dieser Collection.

Die Methoden \code{find(...)} und \code{destroy(...)} sind so Überladen, dass
als ID ein String oder eine \code{BSON::ObjectID} verwendet werden kann.
API-Requests geben die ID als Strings an, in BSON jedoch werden sie als
ObjectIDs gespeichert.

Den Methoden \code{insert(resource)} und \code{update(resource)} wird jeweils
eine Resource übergeben und das BSON-Mapping genutzt um ein BSON-Dokument zu
erstellen beziehungsweise aktualisieren.

Die Methoden \code{all(...)}, \code{find(...)} und \code{find\_by(...)} nutzen
jeweils die Methode \code{execute\_query(...)} für Datenbankabfragen.  Der
MongoDB-Treiber liefert \code{execute\_query(...)} einen Datenbank-Cursor.  Der
Cursor implementiert \code{Iterable} und dadurch die Methode
\code{map(\&block)}, die genutzt wird um die BSON-Dokumente in Objekte vom Typ
\code{T} zu deserialisieren.  Hierfür wird der von BSON-Mapping bereitgestellte
Konstruktor \code{initialize(bson)} genutzt.  Die Methode \code{map(\&block)}
von \code{Iterable} wird \enquote{lazy} ausgeführt, das heißt die
Deserialisierung findet erst statt, wenn über den Iterator iteriert wird.

\subsection{JSON-API Resources}
\label{ssec:ba_json_api_resources}

Zu den in \cref{fig:er_ccc} aufgeführten Entitäten wurden Unterklassen von
\code{JSONApi::Resource} implementiert.  Zusätzlich wurden BSON-Mappings
angegeben die, wie bereits erläutert wurde, von den Repositories genutzt
werden.  \cref{lst:ccc_api_resource_item} Zeigt die Implementierung für die
Resource zur Entität Item und \cref{lst:ccc_api_repository_items} das zugehörige
Repository.  Die restlichen Entitäten wurden ähnlich abgebildet.

\lstinputlisting[%
	float,
	language=Crystal,
	label={lst:ccc_api_resource_item},
	caption={CCCApi: Item-Resource},
]{item_resource.cr}

\lstinputlisting[%
	float,
	language=Crystal,
	label={lst:ccc_api_repository_items},
	caption={CCCApi: Items Repository},
]{items_repository.cr}

