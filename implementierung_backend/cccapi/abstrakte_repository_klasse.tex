\subsection{Abstrakte Repository-Klasse}
\label{ssec:ba_abstrakte_repository_klasse}

Wie bereits erwähnt, wurde in CCCApi vom Repository-Entwurfsmuster Gebrauch
gemacht.  Dazu wurde eine abstrakte Klasse \code{Repository} konzipiert, über
die Datenbankabfragen und -befehle abstrahiert sind.  Hierbei handelt es sich
um eine generische Klasse.  Der generische Typ \code{T} ist wird mit einer
Unterklasse von \code{JSONApi::Resource} belegt.  In
\cref{fig:cccapi_repository} sind die Methoden der Klasse zusammengefasst.

\begin{figure}
	\centering
	\includestandalone{fig_repository_klassendiagramm}
	\caption{CCCApi: Repository}
	\label{fig:cccapi_repository}
\end{figure}

Die Macro-Methode \code{collection\_name} hat den Klassennamen als
Rückgabewert.  Dieser wird genutzt, um im Konstruktor die Instanzvariable
\code{@collection} zu initialisieren.  Alle anderen Methoden arbeiten auf
dieser Collection.

Die Methoden \code{find} und \code{destroy} sind so überladen, dass als ID ein
String oder eine \code{BSON::ObjectID} verwendet werden kann.  API-Requests
geben die ID als Strings an, in BSON jedoch werden sie als ObjectIDs
gespeichert.

Den Methoden \code{insert} und \code{update} wird jeweils eine Resource
übergeben und das BSON-Mapping genutzt, um ein BSON-Dokument zu erstellen
beziehungsweise ein bereits vorhandenes zu aktualisieren.

Die Methoden \code{all}, \code{find} und \code{find\_by} nutzen jeweils die
Methode \code{execute\_query} für Datenbankabfragen.  Der MongoDB-Treiber
liefert für \code{execute\_query} einen Datenbank-Cursor als Rückgabewert.  Der
Cursor implementiert \code{Iterable} und dadurch die Methode
\code{map(\&block)}, die genutzt wird, um die BSON-Dokumente in Objekte vom Typ
\code{T} zu deserialisieren.  Hierfür wird der von BSON-Mapping bereitgestellte
Konstruktor \code{initialize(bson)} genutzt.  Die Methode \code{map(\&block)}
von \code{Iterable} wird \enquote{lazy} ausgeführt, das heißt die
Deserialisierung findet erst statt, wenn der Iterator letztlich durchlaufen
wird.
