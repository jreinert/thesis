\subsection{Route-Matching}
\label{ssec:br_route_matching}

Zum Route-Matching werden aus den Routen-Definitionen reguläre Ausdrücke
erstellt.  Die Syntax der Routen-Definitionen ist durch die Produktionsregeln
in der erweiterten Backus-Naur-Form~\cite{ebnf} in
\cref{lst:crouter_route_syntax} definiert.

\begin{lstlisting}[%
	float,%
	caption = {Crouter: Route-Syntax},%
	label = {lst:crouter_route_syntax}%
]
Route            = StatischerTeil , [ DynamischerTeil ] , [ OptionalerTeil ] ;
StatischerTeil   = '/' , { Zeichenfolge } ;
DynamischerTeil  = Variable , { Gemischt } ;
OptionalerTeil   = '(' , Gemischt , { Gemischt } , [ OptionalerTeil ] , ')' ;
Gemischt         = Zeichenfolge | Variable ;
Zeichenfolge     = Zeichen , { Zeichen } ;
Variable         = ':' ,  VariablenZeichen , { VariablenZeichen } ;
Zeichen          = Grossbuchstabe | Kleinbuchstabe | Ziffer | Sonderzeichen
                 | EscapeSequenz ;
VariablenZeichen = Grossbuchstabe | Kleinbuchstabe | Ziffer | '_' ;
Grossbuchstabe   = 'A' | 'B' | 'C' | 'D' | 'E' | 'F' | 'G' | 'H' | 'I' | 'J'
                 | 'K' | 'L' | 'M' | 'N' | 'O' | 'P' | 'Q' | 'R' | 'S' | 'T'
                 | 'U' | 'V' | 'X' | 'Y' | 'Z' ;
Kleinbuchstabe   = 'a' | 'b' | 'c' | 'd' | 'e' | 'f' | 'g' | 'h' | 'i' | 'j'
                 | 'k' | 'l' | 'm' | 'n' | 'o' | 'p' | 'q' | 'r' | 's' | 't'
                 | 'u' | 'v' | 'x' | 'y' | 'z' ;
Ziffer           = '0' | '1' | '2' | '3' | '4' | '5' | '6' | '7' | '8' | '9' ;
Sonderzeichen    = '\' | '-' | '.' | '_' | '~' | '!' | '$' | '&' | "'" | '*'
                 | '+' | ',' | ';' | '=' | '@' | '/' ;
EscapeSequenz    = '%' , HexZiffer , HexZiffer ;
HexZiffer        = 'A' | 'B' | 'C' | 'D' | 'E' | 'F'
                 | 'a' | 'b' | 'c' | 'd' | 'e' | 'f'
                 | Ziffer ;
\end{lstlisting}

Beim Matching werden die Query-Parameter einer URL nicht berücksichtigt, um für
alle Routen zusätzliche optionale Parameter zu erlauben.  Die Variablen des
dynamischen Teils einer Route werden in \code{params} gesetzt.  Durch beliebig
verschachtelbare optionale Teile können optionale Parameter realisiert werden.

Eine Route gilt dann einem Request als erfolgreich zugeordnet, wenn sowohl
das HTTP-Verb mit dem zur Route gespeicherten Wert als auch die URL mit dem
erzeugten regulären Ausdruck übereinstimmt.
