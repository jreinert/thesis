\subsection{DSL}
\label{ssec:br_dsl}

Die DSL wurde mithilfe von Macros implementiert.  Dadurch kann bereits zur
Compile-Zeit festgestellt werden, ob Routen fehlerhaft definiert sind.  Routen
bestehen aus einem statischen und einem dynamischen Teil.  Der statische Teil
ist ein Substring eines Request-Pfades beginnend mit dem ersten Zeichen.  Der
dynamische Teil ist optional und besteht aus Platzhaltern für variierende Teile
des Pfades.  In Zeile 2 von \cref{lst:crouter_routen} wird eine Route
definiert, die ausschließlich einen statischen Teil besitzt.  Die
Routen-Definition in Zeile 6 hingegen beinhaltet zusätzlich den dynamischen
Teil \code{:id}, der als Platzhalter für eine beliebige Zeichenkette dient.

Um das definieren von typischen RESTful Routes zu erleichtern, können, wie in
Zeile 10 bis 13, Gruppierungen vorgenommen werden.  Routen innerhalb einer
Gruppierung wird der gegebene Präfix vorangestellt, in diesem Fall
\code{pages/}.

Aktionen für Routen können entweder direkt über einen Code-Block wie in Zeile 3
und 7 oder über Methoden einer Controller-Klasse wie in Zeile 11 und 12
ausgeführt werden.  Werden die Aktionen über einen Code-Block ausgeführt, hat
dieser Zugriff auf die Variablen \code{context} und \code{params}.  Über
\code{context} können Request-Header abgefragt oder Response-Header gesetzt
sowie der Response-Body beschrieben werden.  In \code{params} sind die
Parameter eines Requests zusammengetragen.  Die Parameter können aus
Formulardaten im Request-Body oder aus Query-Parametern in der Request-URL
stammen.  Bei Controller-Klassen werden \code{context} und \code{params} dem
Konstruktor übergeben und anschließend die Methode aufgerufen.

\lstinputlisting[%
	language=Crystal,%
	label={lst:crouter_routen},%
	caption={Crouter: DSL},%
	numbers=left,%
	float
]{dsl.cr}
