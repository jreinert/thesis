\section{Crouter}
\label{sec:b_crouter}

Zur Implementierung einer REST-API~\cite{rest} wurde eine Bibliothek benötigt,
die es ermöglicht, Requests anhand des Pfades und HTTP-Verbs bestimmten Aktionen
zuzuordnen.  Hierfür wurde \emph{Crouter} entwickelt.  Crouter stellt eine
DSL\footnote{DSL: Domain specific language} zur Verfügung, mit der diese
Zuordnung vorgenommen werden kann.

In \cref{sec:g_crystal} wurde bereits erwähnt, dass in die Standard-Bibliothek
von Crystal ein HTTP-Server eingebaut ist.  Die eingehenden HTTP-Requests
werden von sogenannten Handlern bearbeitet.  Ein HTTP-Server kann beliebig
viele Handler registrieren, die verkettet voneinander aufgerufen werden.  Zum
Umsetzen eines Handlers wird von der abstrakten Klasse \code{HTTP::Handler}
geerbt und die Methode \code{call(context)} implementiert.  Um die Bearbeitung
des Requests an den nächsten Handler weiterzugeben, kann die Methode
\code{call\_next(context)} aufgerufen werden.

Diese Architektur ermöglicht es, Aufgaben in der Request-Bearbeitung klar
voneinander abzugrenzen und erleichtert es, Teile in projektunabhängige
Bibliotheken auszulagern.  Genau dieser Ansatz wurde mit Crouter verfolgt.  Die
Bibliothek lässt sich in einem beliebigen Crystal-Projekt einbinden und ist
ausschließlich für das Routing zuständig.

\subsection{DSL}
\label{ssec:br-dsl}

\subsection{Route Matching}
\label{ssec:br_route_matching}

\subsection{Performance}
\label{ssec:bj_performance}

Zur Performance-Messung der Serialisierung wurde ein Iterator implementiert,
der zufällige Personen-Resourcen nach dem Schema in
\cref{lst:json_api_resource_macros} erzeugt.  Es wurden Requests mit
unterschiedliche Anzahlen von Resources simuliert, um die Auswirkungen des
Cachings bei steigender Größe der serialisierten Daten zu beobachten.  In
\cref{fig:json_api_performance} sind die Messdaten in einem Diagramm
zusammengefasst.

\begin{figure}
	\center
	\includestandalone{fig_performance}
	\caption{JSON-API: Performance}
	\label{fig:json_api_performance}
\end{figure}

Wie zu erwarten, wird bei größeren Request-Payloads ein proportional größerer
Cache benötigt, da ab einer Payload-Größe von durchschnittlich ungefähr einem
Hundertstel des Caches eine Sättigung erreicht wird und die zum Einbruch der
Response-Geschwindigkeit führt.  Ab dieser Sättigung nimmt das Bereinigen des
Caches von alten Einträgen den Großteil der Rechenleistung in Anspruch.

Aus den Messungen ergibt sich ebenfalls, dass der Cache im Vergleich zu einer
Implementierung ohne ihn für um eine Zehnerpotenz schnellere Request-Zeiten
sorgt.

