\section{Crouter}
\label{sec:b_crouter}

Zur Implementierung einer REST-API~\cite{rest} wurde eine Bibliothek benötigt,
die es ermöglicht Requests anhand des Pfades und HTTP-Verbs bestimmten Aktionen
zuzuordnen.  Hierfür wurde \emph{Crouter} entwickelt.  Crouter stellt eine
DSL\footnote{DSL: Domain specific language} zur Verfügung, mit der diese
Zuordnung vorgenommen werden kann.

In \cref{sec:g_crystal} wurde bereits erwähnt, dass in die Standard-Bibliothek
von Crystal ein HTTP-Server eingebaut ist.  Die eingehenden HTTP-Requests
werden von sogenannten Handlern bearbeitet.  Ein HTTP-Server kann beliebig
viele Handler registrieren, die verkettet voneinander aufgerufen werden.  Zum
umsetzen eines Handlers wird von der abstrakten Klasse \code{HTTP::Handler}
geerbt und die Methode \code{call(context)} implementiert.  Um die Bearbeitung
des Requests an den nächsten Handler weiterzugeben, kann die Methode
\code{call\_next(context)} aufgerufen werden.

Diese Architektur ermöglicht es Aufgaben in der
Request-Beararbeitung klar voneinander Abzugrenzen und erleichtert es Teile in
Projektunabhängige Bibliotheken auszulagern.  Genau dieser Ansatz wurde mit
Crouter verfolgt.  Die Bibliothek lässt sich in einem beliebigen
Crystal-Projekt einbinden und ist ausschließlich für das Routing zuständig.

\subsection{DSL}
\label{ssec:br_dsl}

\subsection{Route Matching}
\label{ssec:br_route_matching}

\subsection{Performance}
\label{ssec:bj-performance}

