\subsection{Performance}
\label{ssec:br_performance}

Um ein möglichst schnelles Routing zu gewährleisten, werden die Statischen
Teile der Routen als Schlüssel einer statischen Hashmap genutzt, in der die
Routen-Definitionen gespeichert werden.  Da mehrere Routen den selben
statischen Teil haben können, werden sie in Arrays als Werte der Hashmap
gespeichert.

Um eine Request-URL einer Route zuzuordnen, wird zunächst die vollständige URL
ohne Query-Parameter als Schlüssel der Hashmap getestet.  Dieser wird solange
von rechts um ein Zeichen gekürzt, bis ein Wert gefunden wird (oder der
Schlüssel keine Zeichen mehr enthält).  Dadurch wird garantiert, dass die
letztlich gefundene Route, den längstmöglichen übereinstimmenden statischen
Teil besitzt.

Erst im nächsten Schritt wird geprüft, ob eine der so bereits eingeschränkten
Routen übereinstimmt.  Damit wird das Matching von RESTful Routes mit
typischerweiser vielen unterschiedlichen statischen Teilen beschleunigt, da
nur ein Bruchteil der definierten Routen mit regulären Ausdrücken geprüft werden
muss.

Das Routing mit Crouter erwies sich als Performant.  In
\cref{tbl:crouter_benchmarks} sind die Benchmark-Ergebnisse zusammengetragen.
Bemerkenswert ist, dass die Response-Zeiten bei steigender Routenanzahl
konstant bleiben.

\begin{table}
\begin{tabu}{X|r|r|r}
	Test & Requests pro Sekunde & Fehlerbereich & Um Faktor langsamer \\
	\hline
	Ohne Router & \num{11.19e3} & $\pm11.00\%$ & $1.00$ \\
	32 Routen & \num{10.43e3} & $\pm10.47\%$ & $1.07$ \\
	64 Routen & \num{10.75e3} & $\pm 9.21\%$ & $1.04$ \\
	128 Routen & \num{10.61e3} & $\pm 9.49\%$ & $1.05$ \\
	256 Routen & \num{10.39e3} & $\pm 9.92\%$ & $1.08$ \\
\end{tabu}
\caption{Crouter: Benchmark-Ergebnisse}
\label{tbl:crouter_benchmarks}
\end{table}
