\subsection{Store}
\label{ssec:fi_store}

Der Teil des Stores, indem die Items gespeichert sind, kann Anhand des
beispielhaften Objekts in \cref{lst:items_store} beschreiben werden.

\lstinputlisting[%
	float,
	language=ES6,
	caption={Items Store},
	label={lst:items_store}
]{store.js}

Die Items-Liste enthält in dem Besipiel drei Einträge.  Der letzte in Zeile
12-15 ist ein Kind vom ersten Item mit der ID \code{'1'}.  Das Feld
\code{selectedItem} beinhaltet das momentan ausgewählte Item im User-Interface.
Ist kein Item ausgewählt, hat das Feld den wert \code{null}.

Der Items-Reducer vereinigt \code{collection},
\code{selectedItem} sowie den in \cref{ssec:fc_store} vorgestellten
\code{components}-Reducer mit der in \cref{ssec:gr_reducer} erläuterten
Funktion \code{combineReducers}.

Der \code{collection}-Reducer nutzt das in \cref{sec:f_collection_handlers}
vorgestellte \code{collectionHanders}-Modul und beinhaltet somit alle zum
Einfügen, Entfernen und Ändern notwendigen Handler.  Zusätzlich wird ein
Handler für die Action \code{FETCH\_ITEMS\_COMPLETE} implementiert.  Diese
Action wird nach einem Login ausgeführt um eine initiale Liste aller Items
zu beziehen und im Store abzulegen.

Der \code{selectedItem}-Reducer registriert Handler für die Aktionen
\code{SELECT\_ITEM\_COMPLETE}, \code{CHANGE\_ITEM},
\code{UPDATE\_ITEM\_COMPLETE}, und \code{REMOVE\_ITEM\_COMPLETE}.  Die
\code{SELECT\_ITEM\_COMPLETE}-Action aktualisiert den State mit dem im Payload
der Action übergebenen Item oder \code{null}.
