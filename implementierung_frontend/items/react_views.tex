\subsection{React-Views}
\label{ssec:fi_react_views}

Für die Navigation der gewählten tief verschachtelten Datenstruktur der Items
wurde ein Menü entwickelt, das jeweils entweder die Wurzeln der Baumstruktur
oder die Kinder des zuletzt ausgewählten Items anzeigt.  Die Navigation der in
\cref{lst:items_store} gezeigten Items ist in \cref{fig:items_menu} dargestellt.

\begin{figure}
	\centering
	\includestandalone{fig_items_menu}
	\caption{Items Menü}
	\label{fig:items_menu}
\end{figure}

Das Menü besteht aus insgesamt vier React-Components: \code{Menu},
\code{MenuLevel}, \code{Item} und \code{MenuLevelItem}.  Die Komponente
\code{Menu} beinhaltet das gesamte Menü und ist als einzige der vier mit dem
Store verbunden.  Aus dem Store werden die States für
\code{items.collection.items} und \code{ui.tree} entnommen.  In \code{ui} sind
User-Interface-Spezifische Informationen, wie die aktuelle Menü-Tiefe,
gespeichert.  Unter dem State-Pfad \code{items.collection.items} befindet sich
die Liste aller Items.

Die \code{Menu}-Komponente implementiert Callback-Funktionen für
User-Interface-Aktionen, die zum Hinzufügen oder Auswählen von Items führen
sollen.  Die \code{render}-Funktion gibt die erste \code{MenuLevel}-Komponente
aus.  Ihr werden die Items, der UI-State, sowie die Callbacks übergeben.

Die Komponente \code{MenuLevel} erwartet einen Pfad, der eindeutig bestimmt,
welchem Baumknoten sie zugeordnet ist.  Der Pfad ist beim ersten Aufruf durch
\code{Menu} leer, beinhaltet jedoch in tieferen Ebenen die IDs aller Vorfahren
des aktuellen Levels bis zur Wurzel.  Die Komponente gibt eine HTML-Liste aus.
Hierzu werden aus den übergebenen Items diejenigen herausgefiltert und
angezeigt, die als Parent-ID das letzte Element des übergebenen Pfades gesetzt
haben, im Falle eines leeren Pfades alle Items mit \code{null} als Parent-ID.

Für jedes so herausgefilterte Item wird ein \code{MenuLevelItem} ausgegeben.
Als Überschrift wird der Name des Parent-Items, oder im Falle des Wurzel-Levels
der String \enquote{Items} erzeugt.  Zusätzlich wird ein Formular mit einem
Textfeld ausgegeben, über das Items hinzugefügt werden können.  Hierfür wird
der von \code{Menu} übergebene Callback genutzt.

In der Komponente \code{MenuLevelItem} wird ein HTML-List-Item erstellt und
darin ein \code{Item} sowie das neue \code{MenuLevel} eingefügt.  Dem
\code{MenuLevel} wird nun der aktuelle Pfad konkateniert mit der ID des Items
übergeben.

Die \code{Item}-Komponente gibt einen Link mit dem Namen des Items aus.  Der
Link löst den in \code{Menu} definierten Callback aus.
