\section{Collection Handlers}
\label{sec:f_collection_handlers}

Im Frontend wurden an zwei Stellen Listen von Daten im Store verwaltet.  Es
stellt sich schnell heraus, dass hier abstrahiert werden kann, damit
Code-Duplizierung vermieden wird.

Das Modul \code{collectionHandlers} kann in Reducer eingebunden werden, über
die eine Listenverwaltung ablaufen soll.  Hierfür muss nur eine Bezeichnung für
einen Listeneintrag angegeben werden.  Anhand dieser Bezeichnung werden Handler
für die Aktionen zum Hinzufügen, Löschen und Ändern von Einträgen erstellt.

\subsection{Hinzufügen von Einträgen}
\label{ssec:fc_hinzufuegen_von_eintraegen}

Für das Hinzufügen von Einträgen werden Handler für die Actions
\code{ADD\_\$\{type\}\_PENDING} und \code{ADD\_\$\{type\}\_COMPLETE}
ergänzt.  Die Variable \code{type} hat den Wert der übergebenen
Bezeichnung für einen Listeneintrag.  Für die Bezeichnung \code{"ITEM"} würden
also Handler für die Aktionen \code{ADD\_ITEM\_PENDING} und
\code{ADD\_ITEM\_COMPLETE} angelegt werden.

Der Action-Creator erzeugt eine \code{ADD\_\$\{type\}} Action, führt einen
\code{create}-Request mit der in \cref{sec:f_api_schnittstelle} beschriebenen
API-Schnittstelle aus und übergibt das zurückgegebene Promise-Objekt als
Payload.  Dadurch teilt die in \cref{ssec:fm_optimistic_updates} beschriebene
\code{optimistic}-Middleware die Action in einen \code{\_PENDING}- und
\code{\_COMPLETE}-Teil.

Mit der \code{\_PENDING}-Action wird der Payload als Eintrag in die Liste
eingefügt.  Zusätzlich wird die von der \code{optimistic}-Middleware
hinzugefügte UUID in den Eintrag eingetragen und ein Feld \code{pending} mit
\code{true} belegt.  So können im View-Layer Vorkehrungen getroffen werden,
um den Eintrag beispielsweise anders darzustellen.

Mit der \code{\_COMPLETE}-Action wird über die UUID der entsprechende Eintrag
aus der Liste herausgesucht, \code{pending} \code{false} gesetzt und die UUID
entfernt.  Schlägt die \code{\_COMPLETE}-Action fehl, wird der Eintrag gelöscht
und somit das Hinzufügen rückgängig gemacht.

Einträge können mit einem Feld \code{pos} sortiert werden.  Dies erhöht jedoch
die Komplexität des Einfügens von Einträgen.  Zunächst muss nämlich die
Position in der Liste gefunden werden, die durch die Sortierung anhand des
\code{pos}-Feldes des Eintrags bestimmt ist.  Hier muss davon ausgegangen
werden, dass die Felder nicht eindeutig sind, da beispielsweise zwei Items, die
unterschiedliche Parent-Items besitzen, gleiche Werte für ihre
\code{pos}-Felder haben können.

\cref{lst:collection_handlers_insert} zeigt die Implementierung dieser
Vorgehensweise.  Es wird die von React bereitgestellte Funktion \code{update}
genutzt, um den State ändern zu können, ohne ihn zu mutieren~\cite{update}.

\lstinputlisting[%
	float,
	language=ES6,
	caption={Collection-Handlers: Einfügen},
	label={lst:collection_handlers_insert}
]{insert.js}

\subsection{Löschen von Einträgen}
\label{ssec:fc_aendern_von_eintraegen}

Um Einträge löschen zu können, erzeugt das Modul Handler für die Actions
\code{REMOVE\_\$\{type\}\_PENDING} und \code{REMOVE\_\$\{type\}\_COMPLETE}.
Analog zu den Actions zum Hinzufügen wird auch hier ein API-Request ausgeführt.
Das daraus resultierende Promise-Objekt wird wieder als Payload für die Action
genutzt und somit von der \code{optimistic}-Middleware behandelt.

Die \code{\_PENDING}-Action löscht einen Listeneintrag mit einem im Payload
übergebenen ID-Feld.  Der Listeneintrag wird jedoch zusätzlich einer separaten
Liste im Store hinzugefügt.

Diese zweite Liste wird genutzt, wenn eine \code{\_COMPLETE}-Action als
fehlgeschlagen markiert ist.  Der Eintrag kann dann wiederhergestellt werden,
indem ein Eintrag mit passender UUID aus der zweiten Liste in die eigentliche
Liste übertragen wird.  Ist die \code{\_COMPLETE}-Action erfolgreich, kann der
Eintrag aus der zweiten Liste entfernt werden.

\subsection{Ändern von Einträgen}
\label{ssec:fc_aendern_von_eintraegen}

Das Modul erstellt die Actions \code{CHANGE\_\$\{type\}},
\code{UPDATE\_\$\{type\}\_PENDING} und \code{UPDATE\_\$\{type\}\_COMPLETE}.
Anders als beim Hinzufügen oder Löschen gibt es hier drei Actions.  Die
\code{CHANGE}-Action wird jedes Mal, wenn ein Nutzer beispielsweise einen
Buchstaben in einem Textfeld eingibt oder löscht, ausgeführt.  Wegen der
Häufigkeit, wird bei der \code{CHANGE}-Action kein API-Request ausgelöst,
sondern nur der Store aktualisiert.  Dies ist nötig, um die Synchronisation von
Store und View zu gewährleisten.

Die \code{UPDATE}-Actions hingegen werden erst ausgelöst, sobald ein Textfeld
den Fokus verliert.  Auf diese Weise kann die Anzahl an nötigen API-Requests
begrenzt werden.  Analog zum Löschen wird hier mit einer zweiten Liste
gearbeitet, in der die Einträge vor den Änderungen gespeichert werden.  So kann
beim Fehlschlagen einer \code{UPDATE\_\$\{type\}\_COMPLETE}-Action der vorige
Status wiederhergestellt werden.

\subsection{Client Synchronisation}
\label{ssec:fc_client_synchronisation}

Alle \code{\_COMPLETE}-Actions werden mit dem Feld \code{broadcast: true} im
Meta-Feld der Action ausgelöst.  Damit werden sie von der
\code{broadcast}-Middleware an die restlichen Clients übermittelt, die ihre
States entsprechend aktualisieren können.
