\subsection{Optimistic Updates}
\label{ssec:fm_optimistic_updates}

Um \emph{optimistic Updates} zu ermöglichen wurde die Middleware
\code{optimistic} implementiert.  Dort wird geprüft, ob der Payload einer Action
ein Promise-Objekt ist.  Ist dies der Fall, wird die Action in zwei Dispatches
aufgeteilt.

Der erste Dispatch erfolgt sofort, doch dem \code{type}-Feld der
Action wird der String \code{"\_PENDING"} angehängt und als Payload der
Wert des \code{meta}-Feldes übergeben.  Zusätzlich wird eine UUID~\cite{RFC4122}
generiert und in dem \code{meta}-Feld der Action eingesetzt.

Sobald das Promise erfüllt ist, wird der zweite Dispatch ausgeführt.  Diesmal
wird dem \code{type}-Feld der String \code{"\_COMPLETE"} angehängt.  Der Payload
wird vom Rückgabewert des Promise bestimmt.  Falls das Promise abgewiesen wird,
enthält der Payload ein Error-Objekt und das Feld \code{error} wird mit
\code{false} beschrieben.  Dem \code{meta}-Feld wird die selbe UUID zugewiesen
wie beim ersten Dispatch.  Dadurch können die Actions eindeutig zugeordnet
werden.  Die UUID ist notwendig um beispielsweise bei fehlgeschlagenen Actions
im Store zu erkennen welches optimistische Update rückgängig gemacht werden
muss.
