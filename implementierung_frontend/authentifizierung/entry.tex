\section{Authentifizierung}
\label{sec:f_authentifizierung}

Zur Authentifizierung wird im State der Applikation ein Boolean-Wert zum
Pfad \code{auth.authenticated} gespeichert.  Der Wert bestimmt, ob der Nutzer
die eigentliche Applikation oder eine Login-View angezeigt bekommt.

Ein Angreifer könnte den Wert manipulieren und sich damit Zugang auf die
Applikation erhoffen.  Dies ist jedoch nicht möglich, da alle API-Requests mit
einem Token authentifiziert werden müssen.  Der Angreifer würde lediglich das
Template der Applikation ohne Inhalt angezeigt bekommen.

Initial ist der Wert von \code{auth.authenticated} \code{false} und der Nutzer
wird zum Einloggen aufgefordert.  Die eingegebenen Login-Daten werden für einen
durch HTTP-Basic-Access~\cite{RFC2617} authentifizierten API-Request genutzt um
einen Login-Token zu bekommen.  Der Request muss über HTTPS erfolgen, um die
Applikation vor MITM-Angriffen\footnote{MITM: Man in the middle} zu schützen.
Ist der Request erfolgreich, wird der Token im Local-Storage~\cite{web-storage}
gespeichert und bei nachfolgenden Requests mitgeliefert.  Wenn sich ein Nutzer
ausloggt, wird der Login-Token von der API gelöscht und im Client aus dem
Local-Storage entfernt.
