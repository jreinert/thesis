\subsection{React und Flux mit Meteor}
\label{ssec:el_react_und_flux_mit_meteor}

Während der Recherche ist eine Softwarearchitektur positiv aufgefallen, die in
\cref{sec:g_react_und_flux} näher betrachtet wird.  Hierbei handelt es sich um
\stichwort{Flux}.  Hierzu soll an dieser Stelle nur erwähnt werden, dass diese
Architektur durch unidirektionalen Datenfluss die Komplexität von Frontends
verringern und das Testen vereinfachen kann.  React hingegen bildet die
View-Ebene des Frontends und reagiert auf Änderungen von durch Flux modifizierte
Daten.  Sowohl Flux als auch React, kann in Meteor eingebunden werden.  Dieser
zu Beginn der Thesis gewählte Ansatz wurde bis in das zweite Drittel der
Bearbeitungszeit weiterverfolgt.  Was letztendlich dazu geführt hat auch diesen
Ansatz teilweise zu verwerfen, wird im Folgenden diskutiert.

Die Vor- und Nachteile dieses Ansatzes sind analog zu \cref{ssec:el_meteor},
wobei zu den Vorteilen zusätzlich die Verbesserung der Testbarkeit und die
fortgeschrittenen Strukturierungsmöglichkeiten, die eine Flux-Architektur mit
sich bringt, gezählt werden können.

\subsubsection{Meteor mit Webpack}
\label{sssec:elf_meteor_mit_webpack}

Wie bereits in \cref{ssec:el_meteor} erläutert, kann es sich Problematisch
gestalten Pakete zu nutzen, die nicht in Atmosphere vertreten sind.  Flux und
React haben viele in der Praxis sinnvolle Erweiterungen, die zum Großteil nur
über npm zu beziehen sind.  Hierzu gehören vor allem Pakete, die das Testen
vereinfachen, indem sie zum Beispiel Mocks zur Verfügung stellen.  Um das
Problem zu umgehen wurde Webpack\footnote{Zu Webpack siehe \cref{sec:g_webpack}}
mit npm eingeführt.  Des weiteren bietet Webpack ein fortgeschrittenes
Abhängigkeitsmanagement, womit das ebenfalls in \cref{ssec:el_meteor} erklärte
Load-Order-Problem gelöst werden konnte.

Trotz der Konfiguration von Webpack als Buildsystem und Abhängigkeitsmanagement
kann nicht vollständig auf das Paketmanagement von Meteor verzichtet werden, da
Meteor-interne Pakete wiederum nicht über npm verfügbar sind.  Dies führt an
mehreren Stellen zu einem Mehraufwand, weil darauf geachtet werden muss, dass
sich Pakete nicht überschneiden, oder weil bestimmte Pakete wie beispielsweise
React nur kompatibel sind, wenn sie über das von Meteor bereitgestellte
Paketmanagement bezogen werden.

\subsubsection{Reaktivität}
\label{sssec:elf_reaktivität}

Im Laufe der Implementierung hat sich herausgestellt, dass das
Reaktivitätskonzept von Meteor nicht mit dem von Flux und React kompatibel ist.
In der Flux-Architektur kann der Datenstamm mit dem die React-Views arbeiten nur
über sogenannte \stichwort{Flux-Actions} verändert werden.  Deshalb mussten
Callbacks eingeführt werden, die bei Änderungen der Meteor-Collections von
Serverseite aus oder ausgehend von weiteren verbundenen Clients entsprechende
Flux-Actions auslösen.

Wenn im Client nun Änderungen an der Datenbank ausgelöst werden sollen, muss
dies ebenfalls über Actions erfolgen.  Gleichzeitig werden aber durch die
Subscription auf die geänderte Collection auch die Callbacks ausgelöst.  Deshalb
muss in den Callbacks separat geprüft werden, ob die Änderung vom Client oder
vom Server beziehungsweise einem anderen Client ausgegangen ist indem jedem
Schreibzugriff eine eindeutige ID mitgegeben wird.  Stimmt diese mit der dem
Client zugewiesenen überein, kann das Auslösen einer (zweiten) Flux-Action
ausgelassen werden.

Durch diese Anpassungen wurde das von Meteor bereitgestellte Reaktivitätskonzept
und damit der größte Vorteil von Meteor nahezu vollständig umgangen.  Dies und
die im vorigen Abschnitt beschriebenen Umstände haben dazu geführt Meteor als
essenziellen Teil der Implementierung zu hinterfragen und letztendlich
mit einer Eigenimplementierung zu ersetzen.
