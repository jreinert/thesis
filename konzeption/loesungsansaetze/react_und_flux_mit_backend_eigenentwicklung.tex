\subsection{React und Flux mit Backend Eigenentwicklung}
\label{ssec:el_react_und_flux_mit_backend_eigenentwicklung}

Die Lösung, die sich letztlich am besten bezüglich der Anforderungen erwiesen
hat, bietet bei größerem Aufwand die volle Kontrolle und die beste Performance.
Hierbei handelt es sich um eine Kombination von React und Flux als Basis einer
puren Frontend-Implementierung mit einer in Crystal\footnote{Crystal wird in
\cref{sec:g_crystal} vorgestellt} umgesetzten JSON-API als pure
Backend-Implementierung.

\subsubsection{Vorteile}
\label{sssec:ele_vorteile}

Durch die klare Trennung von Backend und Frontend wird erfahrungsgemäß eine
bessere Testbarkeit und Wartbarkeit erreicht.  Das System ist dadurch flexibler,
da die Teile ausgetauscht werden können, ohne Änderungen im jeweils anderen Teil
nach sich zu ziehen.

Das API-First-Prinzip wird durch den Ansatz automatisch angewandt und bringt
alle in \cref{ssec:ea_api_first} beschriebenen Vorteile mit sich.  So kann nun
eine statisch kompilierte Programmiersprache im Backend genutzt werden, wodurch
ein Performance-Vorteil gegenüber den anderen Ansätzen erreicht wird.

Als Programmiersprache im Backend wurde Crystal anderen geeigneten Sprachen wie
Go oder Rust vorgezogen, da es von der Syntax Ruby sehr ähnlich ist und damit
den Einstieg für andere Entwickler bei MESO erleichtert.

\subsubsection{Nachteile}
\label{sssec:ele_nachteile}

Als Nachteil erweist sich der durch die Eigenimplementierung entstehende
Mehraufwand.  Dieser relativiert sich jedoch, wenn bedacht wird, dass für die
anderen Lösungsansätze aus Performance-Gründen womöglich eine separate ähnliche
API entwickelt werden müsste.

Der Mehraufwand wird jedoch zusätzlich dadurch erhöht, dass vieles, was in
bekannteren Sprachen als Bibliothek verfügbar wäre, selbst implementiert werden
muss, da Crystal noch sehr jung und unbekannt ist und in der Industrie kaum
vorkommt.  Zusätzlich behält sich die Crystal Community vor, bis zur Version 1.0
Änderungen einzuführen, die zur Folge haben können, dass der Programmcode nicht
mehr mit in der neuen Crystal Version kompiliert werden kann.  Da der Code
jedoch nicht wie bei Skriptsprachen von einem Interpreter abhängig ist, kann das
vorhandene Kompilat problemlos weiter verwendet werden.
