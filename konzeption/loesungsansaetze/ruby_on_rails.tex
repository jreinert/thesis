\subsection{Ruby on Rails}
\label{ssec:kl_ruby_on_rails}

Als ersten Ansatz wurde ein System auf der Basis von Ruby on Rails in Erwägung
gezogen.  Trotz einiger schon aus
\cref{ssec:ep_iooi} deutlich gewordener
Schwächen haben die hier aufgeführten Vorteile dafür gesorgt, den Lösungsansatz
nicht sofort zu verwerfen.  Ein komplett in das Framework integriertes Frontend
wäre aufgrund der Anforderungen in \cref{ssec:ea_reaktives_frontend} nicht in
Frage gekommen.  Ein Backend auf Ruby-on-Rails-Basis kann jedoch von Vorteil
sein.

\subsubsection{Vorteile}
\label{sssec:elr_vorteile}

Die Programmiersprache Ruby~\cite{ruby} wird bei der Firma MESO schon seit vielen
Jahren benutzt und Rails wurde bereits in der Version 1.14 in Projekten
eingesetzt.  Die Einstiegshürde für die restlichen Entwickler in ein
Ruby-on-Rails-basiertes Projekt ist somit deutlich geringer als bei den
nachfolgenden Varianten.

Ruby gehört zudem in der Open-Source-Community zu den bedeutendsten
Programmiersprachen~\cite{ghlangtrends} und gewinnt auch in der Industrie immer
mehr an Verbreitung~\cite{tiobe}.  Daraus lässt sich schließen, dass
Bibliotheken und Frameworks wie Ruby on Rails in Zukunft weiter aktiv entwickelt
werden.

Einen weiteren Vorteil bietet das \stichwort{RubyGems}-Ökosystem
\cite{rubygems}, welches zurzeit 7332 \stichwort{Gems} umfasst.  Gems sind
Programme oder Bibliotheken, die in jedes andere Ruby-Programm eingebunden
werden können.  Somit ist die Wahrscheinlichkeit groß, dass Bibliotheken, die
in weniger verbreiteten Programmiersprachen selbst entwickelt werden müssten,
bereits vorhanden sind und nur eingebunden werden müssen.

\subsubsection{Nachteile}
\label{sssec:elr_nachteile}

Zusätzlich zu den bereits in \cref{ssec:ep_iooi} erläuterten Nachteilen kommen
einige, die als solche nur nach Auffassung des Autors gelten und vom Leser als
Erfahrungswerte mit Ruby on Rails aufgefasst werden sollten.  Zwar ist
scheinbar ein genereller Trend in der Ruby-Community zu ähnlichen
Schlussfolgerungen zu beobachten, doch dies ist schwer nachzuweisen und mag im
Auge des Betrachters liegen.  Der Leser sollte die nachfolgenden
Argumentationen hinterfragen und sich selbst ein Bild über die als Nachteile
angesehenen Eigenschaften machen.

Ein Aspekt von Ruby on Rails, der das Arbeiten an größeren Projekten unnötig
kompliziert gestalten kann, ist die monolithische Bauweise des Frameworks.  Es
muss aber erwähnt werden, dass sich dieser Umstand in den letzten Jahren durch
das Auslagern vieler ehemaliger Kernbestandteile in Gems verbessert hat.
Dennoch benötigt eine Rails-Applikation bereits nach dem Hochfahren mit dem
aktuell proportional zum RAM-Verbrauch leistungsfähigsten Application-Server
Puma~\cite{puma} etwa \MiB{50} Arbeitsspeicher, im Produktiveinsatz meist
sogar das zehnfache oder mehr~\cite{railsappservercomparison}.

Zudem ist Ruby on Rails ein sogenanntes \stichwort{opinionated Framework}.
Opinionated kann mit \emph{starrsinnig} übersetzt werden.  Diese Art von
Frameworks lassen sich durch klare Richtlinien und sinnvolle Standards sehr
effektiv nutzen für \stichwort{Rapid Application Development}, dem schnellen
Anwenden eines prototypischen Vorgehens, um möglichst schnell zu ausführbarem
Code zu gelangen.  Dieser Eigenschaft hat Rails auch einen großen Teil seiner
Beliebtheit zu verdanken.  Die Kehrseite von opinionated Frameworks wird jedoch
erst ersichtlich, wenn versucht wird \enquote{gegen das Framework} zu arbeiten,
das heißt das Framework mit nicht vorgesehenen Funktionalitäten zu erweitern
oder Teile auszulassen.  Dies gestaltet sich oft schwer bis unmöglich.

Als weitere in der Praxis negativ auffallende Eigenschaft von Rails hat
sich die Menge an \stichwort{Metaprogrammierung} in der Code-Basis erwiesen.
Durch Metaprogrammierung ermöglicht man das Ändern von Teilen des Programmcodes
zur Laufzeit. Dies kann in Fehlerfällen zu Seiteneffekten führen, die schwer zu
orten und zu debuggen sind.

Die Nachteile haben aus der Sicht des Autors gegenüber den Vorteilen überwogen
und somit Ruby on Rails als Basis für die Backend-Entwicklung aus weiteren
Überlegungen ausgeschlossen.
