\subsection{Meteor}
\label{ssec:el_meteor}

Eine weitere Überlegung war es die Arbeit mit dem Web-Framework Meteor
umzusetzen.  Meteor gewinnt stetig an Bedeutung~\cite{hotframeworks},
insbesondere seit in Version 1.2 die Unterstützung für React-Views und
ECMAScript 6\footnote{Vorstellung von React und ECMAScript 6 in
\cref{chap:grundlagen}} eingebaut ist~\cite{meteor12}.  Das Framework ist in
NodeJS~\cite{nodejs} implementiert, daher wird zusätzlich zum Frontend auch die
Backend-Programmierung in JavaScript vorgenommen.  Die Vor- und Nachteile von
Meteor sollen nun genauer betrachtet werden.

\subsubsection{Vorteile}
\label{sssec:elm_vorteile}

Ein Attribut von Meteor, das der in \cref{ssec:ea_reaktives_frontend}
aufgeführten Anforderung entgegen kommt, ist die tief im Framework verankerte
Reaktivität.  Meteor implementiert das eigenentwickelte
\stichwort[Meteor]{Distributed Data Protocol} (DDP), was für die
Client-Server-Kommunikation verwendet wird~\cite{ddp}.  Das Protokoll basiert
auf WebSocket und ermöglicht somit einen asynchronen Nachrichtenaustausch
zwischen den Endpunkten.

Besonders interessant dabei ist die Art des Datenbankzugriffes:  Der Client
abonniert eine sogenannte \stichwort[Meteor]{Collection}, die vom Server
freigegeben wurde und bekommt anschließend live Updates über Änderungen am
Datensatz.  Jegliche Modifikationen, die der Client an den Daten vornimmt,
werden lokal an einer im Speicher gehaltenen Kopie ausgeführt.  Anschließend
wird DDP genutzt um die Client- und Serverversionen zu synchronisieren.  Da das
View-Rendering auf Basis der clientseitigen Kopie erfolgt, sind Änderungen
sofort sichtbar und nicht erst nach der verstrichenen Round-Trip-Time zum
Server.  Eine Collection kann durch Datenbankabfragen oder beliebige Filter-
und/oder Transformationsfunktionen eingeschränkt oder verändert
werden~\cite{meteordoc}.

Ein weiterer Vorteil von Meteor ist die eingebaute Unterstützung für das
Deployment auf mobile Endgeräte~\cite{meteormobile} über Apache
Cordova~\cite{cordova}.  Damit können Apps für Android- und iOS-Geräte gepackt
werden, was potenziell Entwicklungsarbeit spart, sollte eine mobile App geplant
sein.  Das war für die Thesis nicht gefordert, somit fiel diese Eigenschaft
nicht weiter ins Gewicht.

\subsubsection{Nachteile}
\label{sssec:elm_nachteile}

Wie aus den obigen Erläuterungen hervor geht, handelt es sich auch bei
Meteor um ein opinionated Framework, in vielerlei Hinsicht sogar in wesentlich
stärkerem Maße als bei Ruby on Rails.  So bietet Meteor zum Beispiel aktuell
immer noch keine Unterstützung für andere Datenbanken als
MongoDB\footnote{MongoDB wird in \cref{sec:g_mongodb} aufgegriffen}.

Ein weiterer negativer Aspekt ist die von Meteor genutzte veraltete
NodeJS-Version\footnote{Aktuell von Meteor genutzte NodeJS-Version: 0.10.40.
Letzte NodeJS-Version: 5.7.1}.  Dies bringt vor allem Probleme mit sich, wenn
ein NPM-Paket\footnote{NPM: Node Package Manager, analog zu RubyGems} genutzt
werden soll, wofür es keine Alternative in den Meteor-eigenen Paketquellen
\emph{Atmosphere}~\cite{atmospherejs} gibt.  Besonders bei Paketen, die native
kompilierte Erweiterungen benötigen, sorgt dieser Umstand für unerwünschte
Seiteneffekte bis hin zur Unbenutzbarkeit des jeweiligen Pakets.

Obwohl Meteor ein opinionated Framework ist, gibt es keine Vorgaben für die
Dateistruktur von Projekten wie es bei Ruby on Rails der Fall ist.  Hinzu kommt,
dass die Load-Order der Dateien nicht beeinflusst werden kann, sondern allein
durch die alphanumerische Sortierung der Dateipfade festgelegt wird.
