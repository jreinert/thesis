\subsection{Vorhandenes Content-Management-System}
\label{ssec:ep_vorhandenes_content_management_system}

Das vorhandene System (IOOI -- kurz für Input-Output-Output-Input) wurde mit
dem Ruby-On-Rails Webframework geschrieben.  Dabei enthält die Anwendung keine
für Rails typische Template-Engine-getriebene View-Schicht, sondern ein mit
Backbone und Marionette \cite{marionette} implementiertes Frontend, das über
API-Requests mit Rails kommuniziert.  Die Art und die Technologien der Umsetzung
haben die folgenden Probleme mit sich gezogen.

\subsubsection{Konflikte bei Änderungen}
\label{sssec:epv_konflikte_bei_aenderungen}

Die zu bearbeitenden Daten sind im Frontend oft zu sehr langen HTML-Formularen
zusammengefasst, was an sich kein Problem darstellt.  Die Änderungen werden
jedoch zusammen in einem Request an den Server übernommen.  Dadurch können
zwei Nutzer niemals gleichzeitig an einem Formular arbeiten, da die Änderungen
des ersten Requests von den des zweiten überschrieben werden würden ohne dass
es dem Nutzer auffällt.

Dem wurde Abhilfe geschafft, indem ein Locking-System implementiert wurde, was
es Nutzern nicht erlaubt gleichzeitig an einem Formular zu Arbeiten.  Da es
aber auf den gleichen Technologien basiert (API-Request für Lock und Unlock)
kann es vorkommen, dass durch einen Netzwerkausfall oder ähnliches niemals ein
Unlock-Request von einem Client gesendet wird und das Item gesperrt bleibt.
Daraufhin wurde ein zusätzliches zeitbasiertes Unlocking eingeführt.  Es wird
jedoch schnell klar, dass eine klassische Request-basierte Client-Server
Konstellation nicht gut geeignet ist um das Problem zu lösen.

\subsubsection{Hoher Wartungsaufwand}
\label{sssec:epv_hoher_wartungsaufwand}

Das Schema für die Daten wurde bewusst sehr strikt konzipiert um konsistente
Datenexporte für die auf die Daten zugreifenden Applikationen zu gewährleisten.
An anderen Stellen hat es jedoch zu massivem Wartungsaufwand geführt, da für
jede Änderung an der Datenstruktur eine Datenbankmigration vonnöten ist.
Durch häufig sogar mehreren Änderungswünschen pro Woche von den Kunden haben
sich zum Zeitpunkt des Verfassens dieser Thesis bereits 337 Migrationen
akkumuliert die alle mit einem global eindeutigen Namen versehen werden mussten.

\subsubsection{Ungenügende Performance}
\label{sssec:epv_ungenuegende_performance}

IOOI arbeitet mit großen Datenmengen beim Exportieren der Daten als JSON an die
Client-Applikationen.  Zur Serialisierung wird die Ruby-Bibliothek
\nohyph{ActiveModelSerializers}~\cite{ams} genutzt.  Diese benötigt jedoch
durchschnittlich \Sec{90} für einen typischen Export, der ein JSON-Dokument von
unter \KiB{500} Größe ausliefert.
