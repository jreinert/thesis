\subsection{API-First}
\label{ssec:ea_api_first}

Wie IOOI, soll auch das neue System nach dem
\stichwort{API-First}-Entwurfsprinzip \cite{api-first} gestaltet werden
(\cref{fig:api_first_architektur}).  Im Gegensatz zur in
\cref{fig:klassische_rails_architektur} dargestellten klassischen
Rails-Architektur wird nur eine serverseitige Applikation benötigt.  Zusätzlich
kann ein einfacher Fileserver zum Ausliefern von nicht serialisierbaren Daten
(\stichwort{Assets}) wie Bildern oder Videos eingesetzt werden.

\begin{figure}[h]
	\centering
	\includestandalone[width=\textwidth]{fig_api_first_architektur}
	\caption{API-First-Architektur}
	\label{fig:api_first_architektur}
\end{figure}

\begin{figure}[h]
	\centering
	\includestandalone[width=\textwidth]{fig_klassische_rails_architektur}
	\caption{Klassische MVC-Architektur in Rails}
	\label{fig:klassische_rails_architektur}
\end{figure}

Im Folgenden wird auf die Vorteile des Entwurfsprinzips eingegangen.

\subsubsection{DRY}
\label{sssec:eaa_dry}

In API-First-Architekturen ist es einfacher das
\stichwort{DRY}-Prinzip\footnote{DRY: Don't repeat yourself} einzuhalten.  Wie
in \cref{fig:api_first_architektur} zu erkennen, muss nur noch in der
API Domain-Logik implementiert werden.  CMS, Webseite und mobile Applikation
greifen alle nur noch auf die API zu, die
\stichwort{CRUD}-Operationen\index{CRUD}\footnote{CRUD: Create, read, update,
delete} auf den Daten abstrahiert.

\subsubsection{Client-Unabhängigkeit}
\label{sssec:eaa_client_unabhaengigkeit}

Durch eine umfangreiche API-Schnittstelle lassen sich beliebige
Client-Applikationen realisieren.  Ein direkter Datenbankzugriff ist nicht
nötig, sodass der Applikationscode auf Clientseite ausgeführt werden kann.  Der
Code kann nun fast vollständig aus View- und Presenter-Logik bestehen, die
ohnehin in jedem Client unterschiedlich ist.

\subsubsection{Geringere Serverlast}
\label{sssec:eaa_geringere_serverlast}

Wie oben erläutert, wird die View- und Presenter-Logik in die Clients
verlagert.  Dadurch muss im Server kein aufwändiges Template-Rendering
betrieben werden, was wiederum die Serverlast verringert.

Ein positiver Nebeneffekt des clientseitigen Template-Renderings ist die nun
eher umsetzbare Implementierung des Servers in einer statisch kompilierten
Programmiersprache, die mangels Flexibilität -- im Vergleich zu Skriptsprachen
-- sonst selten in klassischen, vollständig serverseitigen Web-Applikationen
zum Einsatz kommen würde.
