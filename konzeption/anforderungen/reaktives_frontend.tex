\subsection{Reaktives Frontend}
\label{ssec:ea_reaktives_frontend}

Im Gegensatz zu IOOI soll das neue System über ein \stichwort{reaktives
Frontend} verfügen.  Ein Aspekt von Reaktivität im Web-Kontext ist das
Maskieren von Latenzen, die bei der Kommunikation zwischen Client und Server
auftreten~\cite{reactive-frontends}.  Dies wird durch sogenannte
\stichwort{optimistic Updates} erreicht, auf die in
\cref{ssec:fm_optimistic_updates} näher eingegangen wird.

Oft werden in reaktiven Frontends Technologien wie \stichwort{WebSocket}
\cite{RFC6455} verwendet um einerseits Overhead von HTTP-Requests zu vermeiden
und andererseits serverseitig Events auf den Clients auslösen zu können.  So
lassen sich auch Kommunikationswege zwischen den Clients herstellen.  Diese Art
von Vernetzung vereinfacht die Umsetzung der folgenden weiteren Anforderungen.

\subsubsection{Vorbeugen von Konflikten}
\label{sssec:ear_vorbeugen_von_konflikten}

Das in \cref{ssec:ep_iooi} beschriebene Problem von Konflikten beim
gleichzeitigen Ändern von Dokumenten kann mithilfe von
Client-zu-Client-Kommunikation deutlich eleganter gelöst werden.  So kann
beispielsweise über eine Broadcast-Nachricht allen verbundenen Clients
mitgeteilt werden, dass ein Feld editiert werden soll.  Dieses wird dann von
den restlichen Clients gesperrt.  Somit ist der Server bis auf die
Weiterleitung der Nachrichten an die Clients und das mögliche Behandeln von
Race-Conditions von der Locking-Verwaltung befreit.

\subsubsection{Support in Echtzeit}
\label{sssec:ear_support_in_echtzeit}

Der Betreiber des Systems soll in Echtzeit mit dem Kunden über das Frontend
kommunizieren können.  So sollen zum Beispiel Teile des Interfaces
hervorgehoben, der aktuelle Zustand des Interfaces, wie es der Kunde in dem
Moment vorfindet, synchronisiert, oder über einen Chat weitere Hinweise und
Informationen ausgetauscht werden können.

\subsubsection{Austausch von Informationen und Ideen}
\label{sssec:ear_austausch_von_informationen_und_ideen}

Analog zum Support sollen Nutzer miteinander kommunizieren können.  Insbesondere
soll jederzeit deutlich zu erkennen sein, welcher Nutzer zum aktuellen Zeitpunkt
welche Änderungen durchführt.  Diese Information muss in Echtzeit aktualisiert
werden.
