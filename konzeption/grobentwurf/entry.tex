\section{Grobentwurf}
\label{sec:k_grobentwurf}

In diesem Abschnitt wird der Grobentwurf der Implementierung vorgestellt.  Die
einzelnen Komponenten werden in \cref{chap:implementierung_backend} und
\ref{chap:implementierung_frontend} genauer betrachtet.  Es soll einen Einblick
in das System verschaffen und dem/der Leser/-in damit eine Grundlage für die
späteren Implementierungskapitel vermitteln.

Eine Übersicht des Systems wird in \cref{fig:fmc_ccc} gezeigt.  Mit Clients
sind in dieser Abbildung Applikationen gemeint, die ausschließlich einen
lesenden Zugriff auf die Daten benötigen wie eine Webseite oder
VVVV-Installation.

\begin{figure}
    \centering
    \includestandalone{fig_fmc_ccc}
    \caption{FMC Diagramm: Collaborative Content Creation}
    \label{fig:fmc_ccc}
\end{figure}

Um der Anforderung aus \cref{ssec:ea_reaktives_frontend} gerecht zu werden, muss
das System über einen asynchronen Kommunikationskanal verfügen.  Hierüber können
Clients über Datenänderungen und andere Ereignisse informiert werden.  In
\cref{fig:fmc_async_client_server} wird gezeigt, wie das System aufgebaut ist.
Die Clients sollen hauptsächlich über die REST-API mit dem Server
kommunizieren.  Zusätzlich wird eine WebSocket-Verbindung zum Server aufgebaut,
worüber Broadcast-Nachrichten versendet werden können.  Diese werden vom Server
an die restlichen verbundenen Clients übermittelt.

\begin{figure}
    \centering
    \includestandalone{fig_async_client_server}
    \caption{FMC Diagramm: Asynchrone Client-Server-Kommunikation}
    \label{fig:fmc_async_client_server}
\end{figure}

Die Datenstruktur, die vom System abgebildet wird, ist schlank gestaltet und
lässt sich mithilfe von \cref{fig:er_ccc} beschreiben.

Die zentrale Entität ist das Item.  Dies soll ein generisches Container-Objekt
abstrahieren, beispielsweise eine Unterseite einer Webseite.  Ein Item kann
rekursiv beliebig viele weitere Items enthalten.  Damit wird eine Baumstruktur
gebildet, womit sich die meisten Webseiten oder Apps modellieren lassen.

Items können des Weiteren beliebig viele Components enthalten.  Diese Entität
bildet verschiedene Inhalte ab.  Das type-Attribut bestimmt die Art des
Inhaltes. So unterscheidet es sich beispielsweise bei einzeiligem oder
mehrzeiligem Text, bei booleschen Inhalten oder Datei-Uploads.  Das
value-Attribut enthält eine String-Repräsentation des Inhaltes.

Zusätzlich wird zur Benutzerverwaltung eine User-Entität definiert.  Zu jedem
User wird das gehashte Passwort in der Datenbank gespeichert.  Zusätzlich
gehören beliebig viele Tokens zu einem User.  Diese werden genutzt um
API-Requests zu authentifizieren.

\begin{figure}
    \centering
    \includestandalone{fig_er_ccc}
    \caption{ER Diagramm: Collaborative Content Creation}
    \label{fig:er_ccc}
\end{figure}
