\subsection{Syntax}
\label{ssec:gc_syntax}

In diesem Abschnitt wird kurz auf die Eigenheiten der Syntax von Crystal
eingegangen um ein Grundverständnis für die in
\cref{chap:implementierung_backend} aufgeführten Code-Listings voraussetzen zu
können.

\subsubsection{Vergleich mit Ruby}
\label{sssec:gcs_vergleich_zu_ruby}

Wir betrachten das Beispiel aus \cref{lst:cr_http_server}.  Ruby hat zwar keinen
in der Standardbibliothek eingebauten HTTP-Server, doch abgesehen davon würde
der Code vom Ruby-Interpreter verstanden werden.  Oft können kleinere
Ruby-Scripte ohne oder mit nur geringem manuellen Eingreifen von Crystal
kompiliert werden.  Ein wichtiger Unterschied ist jedoch, dass das in Ruby als
String-Trennzeichen\footnote{Zusätzlich zu den doppelten Anführungsstrichen
\texttt{"}} genutzte Hochkomma \texttt{'} in Crystal als Trennzeichen für
Character-Literale verwendet wird.  Somit steht \code{'a'} in Crystal für das
Unicode-Zeichen \code{'a'} und die in Ruby syntaktisch korrekte Zeichenfolge
\code{'test'} würde einen Compiler-Fehler produzieren.

\lstinputlisting[%
    caption={Ein einfacher HTTP-Server},%
    label={lst:cr_http_server},%
    language=Crystal,%
    float%
]{http_server.cr}

\subsubsection{Objektorientierung}
\label{sssec:gcs_objektorientierung}

Anhand der folgenden Beispiele soll die Funktionsweise von Vererbung und
Mixins in Crystal erläutert werden.  Im Gegensatz zu Ruby unterstützt
Crystal abstrakte Klassen und Methoden.  In \cref{lst:cr_animal} ist eine
abstrakte Klasse \code{Animal} definiert.  Mit der Methode \code{initialize}
wird der Konstruktor einer Klasse definiert.  Variablen die mit \code{@}
beginnen werden als Instanzvariablen angelegt.  Wird eine Instanzvariable nicht
im Konstruktor initialisiert so gilt sie als \stichwort{nilable}.  Die Bedeutung
dieses Begriffs wird in \cref{ssec:gc_typensystem} erläutert.  In Zeile 2 wird
ein Getter für die Instanzvariable \code{@alive} deklariert.  Dabei sind die
Formen \code{getter} \code{getter?} oder \code{getter!} zulässig.  Deren
Unterschiede werden hier nicht weiter behandelt, können aber in der offiziellen
Dokumentation von Crystal~\cite{crystaldoc} nachgeschlagen werden.

\lstinputlisting[
    caption={Objektorientierung (1)},
    label={lst:cr_animal},
    language=Crystal,%
    float
]{animal.cr}

Abstrakte Methoden wie in Zeile 22 werden ohne Rumpf mit dem Schlüsselwort
\code{abstract} gekennzeichnet.  Wenn es versäumt wird, die Methode in einer
nicht abstrakten Unterklasse zu implementieren, wird ein Compilerfehler
produziert.

Das in Zeile 25 genutzte Schlüsselwort \code{self} verweist auf die Instanz
der Klasse.  In den Zeilen 30 und 35 wird vom Schlüsselwort \code{super}
Gebrauch gemacht.  Dies verweist auf die Instanz der Oberklasse.  Wird sie wie
hier ohne Zusätze verwendet, wird die Implementierung der jeweiligen Methode in
der Oberklasse \emph{mit den selben Parametern} aufgerufen.  Sollen Parameter
ausgelassen werden, muss die geklammerte Variante
\code{super(parameter1, parameter2, ...)} verwendet werden.

Ab Zeile 28 wird das Modul \code{Carnivore} definiert.  Ein Modul kann sowohl
ausimplementierte, als auch abstrakte Methoden enthalten.  Es können zusätzlich
zu neuen Methoden auch existierende überschrieben -- Zeile 29 -- und/oder
überladen werden -- Zeile 35.  Die Überladung muss den Typ von mindestens einem
Parameter anders einschränken.  Eine Typeinschränkung geschieht indem ein
Variablenname gefolgt wird von einem Doppelpunkt und dem Typ.
Typeinschränkungen sind obligatorisch bei Arrays oder Hashes (Maps).

In den Zeilen 9 und 18 werden Exceptions geworfen.  Hier kann entweder ein
String oder eine Instanz der Klasse Exception oder ihrer Unterklassen übergeben
werden.  Zu beachten ist hierbei die \emph{in-line}-Schreibweise des
Bedingungsblocks, die bei kurzen, einzeiligen Blöcken oder Guard-Clauses die
Lesbarkeit verbessert.

In \cref{lst:cr_extend_include} werden die Klassen \code{Mouse} und \code{Snake}
implementiert.  Beide Klassen erben von \code{Animal}.  Dies wird mit
\code{< Oberklasse} gekennzeichnet.  Die Klasse \code{Snake} wird in Zeile 13
zusätzlich mithilfe des Moduls \code{Carnivore} erweitert.

\lstinputlisting[
    caption={Objektorientierung (2)},
    label={lst:cr_extend_include},
    language=Crystal,%
    float%
]{extend_include.cr}

In der Klasse \code{Mouse} wurde die abstrakte Methode \code{movement} nicht
explizit implementiert, sondern mit einem Getter realisiert.  In Wirklichkeit ist
das Schlüsselwort \code{getter} nur ein Macro, das eine Methode mit dem
angegebenen Namen definiert, und den aktuellen Wert der gleichnamigen
Instanzvariablen zurückgibt.

\subsubsection{Generics}
\label{sssec:gsc_generics}

Crystal erlaubt das Definieren von generischen Klassen und Modulen.  Die Klasse
\code{Array} ist beispielsweise eine generische Klasse, die wiederum das
generische Modul \code{Enumerable} einbindet.

\cref{lst:cr_generics} zeigt eine einfache Binärbaumimplementierung in Crystal.
Die Klasse \code{BinaryTree} ist durch das geklammerte \texttt{T} als generisch
gekennzeichnet.  Auf die Implementierung wird im nächsten Abschnitt eingegangen,
da sie Aufgrund von einigen syntaktischen Spezialitäten einer Erläuterung
bedarf.

Zunächst betrachten wir die Instantiierung eines Binärbaums.  Es fällt auf, dass
nirgends der generische Typ von \code{BinaryTree} angegeben wurde.  Dennoch
kompiliert das Programm und gibt die dem Baum hinzugefügten Elemente in der
korrekt sortierten Reihenfolge aus.  Das liegt daran, dass der Typ des
Parameters für den Konstruktor mit \code{Comparable(T)} eingeschränkt wurde.  Da
in Zeile 23 der Binärbaum mit einem Parameter vom Typ \code{String}
initialisiert wird und \code{String} wiederum \code{Comparable(String)}
einbindet, kann der generische Typ eindeutig mit \code{String} substituiert
werden.

\lstinputlisting[
    caption={Generics und Code-Blocks},
    label={lst:cr_generics},
    language=Crystal,%
    float%
]{generics.cr}

\subsubsection{Code-Blocks}
\label{sssec:gsc_code_blocks}

In Crystal wird sehr häufig mit sogenannten \stichwort[Crystal]{Code-Blocks}
gearbeitet, deren Erläuterung essenziell zum Verständnis der in
\cref{chap:implementierung_backend} folgenden Code-Listings sind.  Code-Blöcke
entsprechen den aus der funktionalen Programmierung bekannten Lambdas mit dem
Unterschied, dass sie nicht aus ihrem Kontext returnen können.

Ein Code-Block wird übergeben indem der Methodenaufruf, der einen Code-Block
erwartet, entweder von einer öffnenden geschweiften Klammer, oder dem
Schlüsselwort \code{do} gefolgt wird.  Nachfolgend werden mögliche an den
Code-Block übergebene Parameter zwischen zwei Pipe-Symbolen benannt.  Ein
Beispiel hierfür findet sich in Zeile 28.

Eine Methode, die einen Code-Block annimmt kann diesen mit dem Schlüsselwort
\code{yield} aufrufen.  Analog zu Methodenaufrufen können hier Parameter
übergeben werden.

Code-Blocks können in Methoden auch \enquote{gefangen} und weitergegeben werden.
Wird dies getan, wandelt der Compiler den Code-Block in eine anonyme Funktion
um.  Dies hat zur Folge, dass es beispielsweise nicht mehr möglich ist aus
Schleifen im umliegenden Kontext herauszuspringen.  Wird ein Code-Block
gefangen, müssen dessen Parameter typisiert werden, um eine anonyme Funktion
korrekt generieren zu können.

Die Methode \code{each} fängt den übergebenen Code-Block mit der in Zeile 16
verwendeten Syntax.  Dem Block-Parameter wird der generische Typ T
zugeordnet.

In den Zeilen 17 und 19 wird der Code-Block nun den Aufrufen der Methode
\code{each} im linken und rechten Ast mitgegeben.  Die Syntax
\code{methode(\&.zweite\_methode)} kann bei Methodenaufrufen verwendet werden, die
einen Code-Block erwarten, um den semantisch äquivalenten Ausdruck
\code{methode \{ |var| var.zweite\_methode \}} abzukürzen.

Zuletzt soll der Vollständigkeit halber noch \code{try} erläutert werden.  Die
Methode \code{try} ist für jedes Objekt definiert, erwartet einen Code-Block und
übergibt diesem die Instanz, auf der die Methode aufgerufen wurde.  Als
Rückgabewert gibt die Methode den des Codeblocks aus.  Eine Ausnahme bilden
jedoch die Klassen \code{Nil}, deren Instanzen einem \code{null} in den meisten
anderen Programmiersprachen entsprechen und die Klasse \code{Boolean}.  Diese
Klassen definieren \code{try} so, dass der übergebene Code-Block im Falle von
\code{Nil} nie -- und von \code{Boolean} nur beim Wert \code{true} aufgerufen
wird.  Wird der Code-Block nicht aufgerufen, hat \code{try} den Rückgabewert
\code{nil}, also eine \code{Nil}-Instanz.  Dieses Beispiel demonstriert auch wie
Konsequent die Objektorientierung im Sprachdesign von  Crystal umgesetzt wird.
