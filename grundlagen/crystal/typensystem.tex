\subsection{Typensystem}
\label{ssec:gc_typensystem}

Trotz der strengen Typisierung, die es ermöglicht viele Programmierfehler
bereits zur Compilezeit zu erkennen, können Crystal-Programme, wie in
\cref{ssec:gc_syntax} gezeigt wurde, sehr schlank und gut lesbar sein.  Weshalb
das gelingt und in Crystal nicht -- wie in den meisten streng typisierten
Sprachen -- jede Variable und der Rückgabewert jeder Methode vom Entwickler mit
einem Typen definiert werden muss, soll hier kurz erläutert werden.

\subsubsection{Union-Types}
\label{sssec:gct_union_types}

Wir betrachten das Crystal-Programm aus \cref{lst:runtime-type-checks}.  Das
Kompilieren würde mit der Meldung in Zeile 13 fehlschlagen, wäre die Zeile
darüber nicht auskommentiert.  Der Compiler analysiert also den Code und
erkennt, dass die Variable \code{sheep} auf eine Instanz vom Typ \code{Wolf}
verweisen kann.  Die Variable hat den Typ \code{Sheep | Wolf}.  Hierbei handelt
es sich um einen sogenannten \stichwort[Crystal]{Union-Type}.  Union-Types
ermöglichen es Arrays und andere generische Container-Klassen mit Instanzen
unterschiedlicher Typen zu befüllen, der Union-Type muss jedoch beim
Instanziieren einer solchen Klasse bekannt sein.  Es können beispielsweise
keine Werte vom Typ \code{Int32} einem \code{Array(String | Char)} hinzugefügt
werden.

Variablen können durch geeignete Bedingungsblöcke in ihren Typen reduziert
werden.  Durch die Bedingung in Zeile 14 wird der Typ von \code{sheep} innerhalb
des Blocks auf \code{Sheep} reduziert, sodass der Methodenaufruf in
Zeile 15 keinen Compilezeit-Fehler auslöst.

Besonders hilfreich ist diese Eigenschaft zum Aufspüren von Fehlern, die in
vielen anderen Programmiersprachen als Null-Pointer-Exceptions bekannt sind.  Da
\code{nil} auch nur eine Instanz vom Typ \code{Nil} ist, können diese Fehler
analog zu Zeile 12 in \cref{lst:runtime-type-checks} bereits zur Compilezeit erkannt
werden.

\lstinputlisting[
    caption={Runtime Type-Checks},
    label={lst:runtime-type-checks},
    language=Crystal,%
    float%
]{runtime_type_checks.cr}

\subsubsection{Nilable}
\label{sssec:gct_nilable}

In \cref{ssec:gc_syntax} wurde die Eigenschaft \stichwort[Crystal]{nilable} von
Instanzvariablen angesprochen.  Diese Eigenschaft wird vom Compiler vergeben,
wenn die Variable nicht im Konstruktor mit einem Wert ungleich \code{nil}
initialisiert wurde, oder wenn die Variable während der Laufzeit den Wert
\code{nil} zugewiesen bekommen könnte.

Ist eine Instanzvariable vom Compiler als nilable deklariert, kann nur auf die
Schnittmenge der Methoden von \code{Nil} und dem alternativen Typ zugegriffen
werden.  Um Methoden des anderen Typs aufzurufen, muss der Compilezeit-Typ
ähnlich wie in Zeile 14 in \cref{lst:runtime-type-checks} auf den korrekten
Typen reduziert werden.  Hierbei genügt es jedoch nicht die Variable zu prüfen,
da sie sich im Bedingungsblock bereits wieder geändert haben könnte.  Sie muss
stattdessen einer lokalen Variablen zugewiesen werden, die anschließend geprüft
oder innerhalb eines \code{try} Code-Blockes genutzt wird.

Eine nilable Variable ist genaugenommen vom Union-Type des eigentlichen Typen
und \code{Nil}.  Da nilable Variablen sehr häufig vorkommen, existiert eine
Kurzschreibweise: Durch das Anhängen eines Fragezeichens kann einem beliebigen
Typ \code{Nil} hinzugefügt werden.  Somit sind die beiden Typen
\code{String | Nil} und \code{String?} semantisch äquivalent.
