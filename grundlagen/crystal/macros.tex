\subsection{Macros}
\label{ssec:gc_macros}

Da Crystal keine Metaprogrammierung zur Laufzeit unterstützt, besitzt es ein
umfangreiches Macrosystem, mit dem viele durch Metaprogrammierung erreichte
Funktionen nachempfunden werden können.  Innerhalb vom Macro-Kontext kann auf
den \stichwort[Crystal]{AST} \footnote{Abstract Syntax Tree} zugegriffen
und daran Änderungen vorgenommen werden.

In \cref{lst:cr_macro} wird ein einfaches Macro definiert und aufgerufen.
Der Methodenaufruf von \code{id} auf der Macrovariablen \code{name} in Zeile 2
liefert die id der Variablen im AST.  Der Aufruf wird genutzt um String- oder
Symbol-Trennzeichen aus dem Wert zu entfernen

\lstinputlisting[%
    caption={Crystal Macro (1)},%
    label={lst:cr_macro},%
    language=Crystal,%
    float%
]{macro.cr}


Die Syntax von Macros ist der von Twig~\cite{twig} -- einer Template-Engine für
PHP -- ähnlich.  Ausdrücke innerhalb von doppelt geschweiften Klammern werden
ausgewertet und an der Stelle in den Code eingefügt.  Werden Ausdrücke gefolgt
von einem Prozentzeichen innerhalb von einfach geschweiften Klammern angegeben,
werden diese zwar ausgewertet, jedoch nicht in den Code eingesetzt.  Der
Compiler ersetzt den Aufruf im Code mit der in \cref{lst:cr_expanded_macro}
aufgeführten Methode.

\lstinputlisting[%
    caption={Crystal Macro (2)},%
    label={lst:cr_expanded_macro},%
    language=Crystal,%
    float%
]{expanded_macro.cr}

Crystal unterstützt zusätzlich verschachtelte Macros:  Hier müssen die im $(n +
1)$-ten Durchgang auszuwertenden Ausdrücke mit $n$ vorangestellten
Backslash-Zeichen versehen werden.
