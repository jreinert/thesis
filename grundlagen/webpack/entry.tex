\section{Webpack}
\label{sec:g_webpack}

Webpack ist eine für NodeJS geschriebene Software zum Verwalten von größeren
JavaScript-Anwendungen~\cite{webpack}.  Zum Verringern von Ladezeiten werden in
modernen Webanwendungen sogenannte \stichwort{statische Assets} wie JavaScript,
CSS und Bilddateien üblicherweise zu jeweils einer Datei zusammengefügt -- mit
Ausnahme von Bildern -- und komprimiert.  Dieser Prozess wird von Webpack
automatisiert.

Zum Entwickeln kann ein von Webpack bereitgestellter Server genutzt werden, der
bei Änderungen der Quelldateien den Build-Prozess automatisch ausführt.
Dies wird durch einen hierarchischen Cache beschleunigt.

\subsection{Loader}
\label{sec:gw_loader}

Webpack stellt mit sogenannten \emph{Loadern}\index{Webpack!Loader} ein
Plugin-System zur Verfügung, über das beliebige Transformationen der Quellen
oder das Hinzufügen von Inhalten erfolgen können.

In CCC wurden beispielsweise Loader genutzt, um ECMAScript 6 in
JavaScript~\cite{babelLoader} sowie SASS in CSS~\cite{sassLoader} zu transpilen.

\subsection{Auflösen von Abhängigkeiten}
\label{sec:gw-aufloesen-von-abhaengigkeiten}

