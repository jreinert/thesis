\subsection{Objektorientierung}
\label{ssec:ge_objektorientierung}

Zusätzlich zu den im letzten Abschnitt aufgeführten Änderungen, wurde mit der
sechsten Edition von ECMAScript das Paradigma der Objektorientierung in das
Sprachdesign eingebaut.  So können nun mit dem Schlüsselwort \code{class}
Klassen definiert, mit \code{extends} Vererbung genutzt und mit \code{super} auf
die Oberklasse zugegriffen werden wie in \cref{lst:es6_oop} demonstriert wird.

\lstinputlisting[%
    caption={Objektorientierung in ES6},%
    label={lst:es6_oop},%
    language=es6,%
    float
]{oop.js}
