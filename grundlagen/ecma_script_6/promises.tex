\subsection{Promises}
\label{ssec:ge_promises}

Das in ES 6 eingeführte Konzept von \stichwort{Promises}~\cite{promises}
vereinfacht den Umgang mit asynchronen Aktionen wie beispielsweise
AJAX-Requests.  Promises sind \enquote{Versprechen}, die zeitversetzt
erfüllt oder abgewiesen werden können.  Ein Promise-Objekt hat die
Methoden \code{then(...)} und \code{catch(...)} die jeweils eine Funktion
übergeben bekommen.  Die der Methode \code{then(...)} übergebene Funktion
wird mit beim erfolgreichen erfüllen des Promises aufgerufen.  Im Falle
einer Exception oder dem expliziten Abweisen des Promises wird die
Funktion aufgerufen, die \code{catch(...)} übergeben wurde.

Promises können verkettet werden, indem aus einer \code{then(...)} übergebenen
Funktion wieder ein Promise als Rückgabewert ausgegeben wird.  In
\cref{lst:es6_promises} wird der Umgang mit Promises anhand eines Beispiels
erläutert.

\lstinputlisting[%
	float,
	language=ES6,
	label={lst:es6_promises},
	caption={Promises}
]{promises.js}
