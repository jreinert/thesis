\subsection{Destructuring}
\label{ssec:ge_destructuring}

Pattern-Matching ist ein vor allem aus funktionalen Programmiersprachen
bekanntes Konzept.  In ECMAScript 6 wurde es in Form von Destructuring von
Objekten eingeführt.  Diese Technik soll nun erläutert werden.

Gegeben sei das verschachtelte Objekt \code{person} aus
\cref{lst:es6_destructuring_1}.  Es sei angenommen, dass im nachfolgenden Code
mehrmals auf verschiedene Attribute des Objekts zugegriffen werden soll.
Es liegt nahe diese Attribute in Variablen zwischenzuspeichern.

\lstinputlisting[%
    caption={Destructuring (1)},%
    label={lst:es6_destructuring_1},%
    language=es6,%
    float
]{destructuring_1.js}

Dieser Prozess wird durch Destructuring um ein Vielfaches verkürzt.  In
\cref{lst:es6_destructuring_2} werden die Attribute \code{id}, \code{name} und
\code{address.street} den Variablen \code{id}, \code{name} und \code{street}
zugewiesen.  In \cref{lst:es6_destructuring_3} wird zusätzlich von der in
\cref{lst:es6_objects} vorgestellten Kurzschreibweise Gebrauch gemacht.

\lstinputlisting[%
    caption={Destructuring (2)},%
    label={lst:es6_destructuring_2},%
    language=es6,%
    numbers=none,%
    float
]{destructuring_2.js}

\lstinputlisting[%
    caption={Destructuring (3)},%
    label={lst:es6_destructuring_3},%
    language=es6,%
    numbers=none,%
    float
]{destructuring_3.js}

Destructuring kann auch in Funktionsparametern genutzt werden um aus übergebenen
Objekten bestimmte Attribute variablen zuzuweisen, wie in
\cref{lst:es6_destructuring_von_funktionsparametern} demonstriert wird.

\lstinputlisting[%
    caption={Destructuring von Funktionsparametern},%
    label={lst:es6_destructuring_von_funktionsparametern},%
    language=es6,%
    numbers=none,%
    float
]{destructuring_fkt.js}
