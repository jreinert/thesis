\subsection{Syntax}
\label{ssec:ge_syntax}

Die Syntax von ES6 im Vergleich zu ES5 hat sich zwar durch Einführen von neuen
Schlüsselwörtern und Schreibweisen verändert, ist jedoch rückwärtskompatibel zu
ES5.  In diesem Abschnitt sollen nun die zum Verständnis der Implementierung
dieser Thesis wichtigsten Änderungen vorgestellt werden.

\subsubsection{Die Schlüsselwörter \texttt{let} und \texttt{const}}
\label{sssec:ges_let_und_const}

Variablen werden in JavaScript bekanntlich mit dem Schlüsselwort \code{var}
deklariert.  In ES6 wurden die Schlüsselwörter \code{let}\index{ECMAScript
6!\texttt{let}} und \code{const}\index{ECMAScript 6!\texttt{const}} eingeführt.
Wird eine gewöhnliche Variable benötigt, ist \code{let} dem Variablennamen
voranzustellen.  Dies entspricht dem Schlüsselwort \code{var} aus ES5 mit dem
Unterschied, dass \code{let}-Variablen nur innerhalb ihres Scopes sichtbar
sind.

Betrachten wir das Beispiel in \cref{lst:es6_let}.  Es werden auf zwei Arten
Variablen deklariert.  In der ersten Hälfte mit \code{var} und in der zweiten mit
\code{let}.  In beiden Fällen ist die erste, auf oberster Ebene eingeführte
Variable wie erwartet definiert.  Die zweite, innerhalb eines Bedingungsblocks
definierte Variable ist jedoch nur mit dem Schlüsselwort \code{var} im
darüberliegenden Scope sichtbar.  Aus diesem Grund ist \code{let} \code{var} in
jedem Fall zu bevorzugen.

\lstinputlisting[%
    caption={Das Schlüsselwort \code{let}},%
    label={lst:es6_let},%
    language=es6,%
    float
]{let.js}

Auf eine Variable, der das Schlüsselwort \code{const} vorangestellt ist, kann
analog zu \code{let} nur im Scope, in dem es deklariert wurde, zugegriffen
werden.  Zusätzlich kann sie nur einmal initialisiert werden, wie aus
\cref{lst:es6_const} hervorgeht.

\lstinputlisting[%
    caption={Das Schlüsselwort \code{const}},%
    label={lst:es6_const},%
    language=es6,%
    float
]{const.js}

\subsubsection{Kurzschreibweisen}
\label{sssec:ges_kurzschreibweisen}

In ES6 wurden einige Kurzzschreibweisen eingeführt, von denen zwei kurz
vorgestellt werden sollen.  Sogenannte \stichwort[ECMAScript
6]{Arrow-Functions} sind anonyme Funktionen in einer in
\cref{lst:es6_arrow_functions} gezeigten neu eingeführten Kurzschreibweise.  Es
fällt zunächst auf, dass keine Return-Statements nötig sind.  Dies ist immer
dann der Fall, wenn der Funktionskörper aus einem einzigen Ausdruck besteht.
Dieser wird dann als Rückgabewert ausgewertet.  Besteht der Funktionskörper aus
mehreren Äusdrücken, müssen diese mit geschweiften Klammern umschlossen werden
und ein Return-Statement vorhanden sein.  Sollen der Funktion mehrere Parameter
übergeben werden, müssen diese mit runden Klammern umschlossen werden.  Um ein
Objekt als einzigen Ausdruck im Funktionskörper nutzen zu können, muss dieses
ebenfalls mit runden Klammern umschlossen werden.

\lstinputlisting[%
    caption={Arrow-Functions},%
    label={lst:es6_arrow_functions},%
    language=es6,%
    float
]{arrow_functions.js}

Für Objekte wurde ebenfalls eine Kurzzschreibweise eingeführt. Diese wird in
\cref{lst:es6_objects} gezeigt.  Falls eine Referenz mit dem selben Namen wie
der Objektschlüssel im aktuellen Scope existiert, kann die übliche Zuweisung mit
dem Doppelpunkt ausgelassen werden.

\lstinputlisting[%
    caption={Kurzschreibweise für Objekte},%
    label={lst:es6_objects},%
    language=es6,%
    float
]{objects.js}

\subsection{String Interpolation}
\label{sssec:ges_string_interpolation}

In der Implementierung wurde an vielen Stellen von der durch ES6 eingeführten
String-Iterpolation Gebrauch gemacht.  Soll ein String interpoliert werden, muss
er mit den Trennzeichen \code{`} eingeleitet und beendet werden.  Ausdrücke im
String, die von \code{\$\{} und \code{\}} umschlossen sind, werden ausgewertet
und ersetzt.
