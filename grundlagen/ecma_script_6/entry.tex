\section{ECMAScript 6}
\label{sec:g_ecma_script_6}

ECMAScript 6, kurz ES6 oder ES 2015, ist die sechste Edition der ECMAScript
Spezifikation~\cite{es6} auf dessen Grundlage JavaScript beruht.  Die momentan
von Webbrowsern unterstützte Edition ist ECMAScript 5.  Um also ECMAScript 6 in
der Entwicklung nutzen zu können, wird ein sogenannter \stichwort{Transpiler}
benötigt.  Dieser wandelt eine höhere Ausgangssprache -- hier ES6 -- in eine
weitere höhere Zielsprache um, in diesem Fall ES5 beziehungsweise JavaScript.
Zu dem Transpiling-Prozess in der Entwicklung wird in \cref{sec:g_webpack} Bezug
genommen.

Diese Edition führte größere Änderungen an der Syntax ein und erweiterte die
Spezifikation mit programmiertechnischen Verfahren wie Pattern-Matching und dem
Paradigma der Objektorientierung.  Diese Änderungen erweitern die Möglichkeiten
der Strukturierung von Frontend-Code und begründeten die Auswahl von ES6 als
Implementierungssprache im Frontend.

\subsection{Syntax}
\label{ssec:ge-syntax}

\subsection{Destructuring}
\label{ssec:ge_destructuring}

Pattern-Matching ist ein vor allem aus funktionalen Programmiersprachen
bekanntes Konzept.  In ECMAScript 6 wurde es in Form von Destructuring von
Objekten eingeführt.  Diese Technik soll nun erläutert werden.

Gegeben sei das verschachtelte Objekt \code{person} aus
\cref{lst:es6_destructuring_1}.  Es sei angenommen, dass im nachfolgenden Code
mehrmals auf verschiedene Attribute des Objekts zugegriffen werden soll.
Es liegt nahe diese Attribute in Variablen zwischenzuspeichern.

\lstinputlisting[%
    caption={Destructuring (1)},%
    label={lst:es6_destructuring_1},%
    language=es6,%
    float
]{destructuring_1.js}

Dieser Prozess wird durch Destructuring um ein Vielfaches verkürzt.  In
\cref{lst:es6_destructuring_2} werden die Attribute \code{id}, \code{name} und
\code{address.street} den Variablen \code{id}, \code{name} und \code{street}
zugewiesen.  In \cref{lst:es6_destructuring_3} wird zusätzlich von der in
\cref{lst:es6_objects} vorgestellten Kurzschreibweise Gebrauch gemacht.

\lstinputlisting[%
    caption={Destructuring (2)},%
    label={lst:es6_destructuring_2},%
    language=es6,%
    numbers=none,%
    float
]{destructuring_2.js}

\lstinputlisting[%
    caption={Destructuring (3)},%
    label={lst:es6_destructuring_3},%
    language=es6,%
    numbers=none,%
    float
]{destructuring_3.js}

Destructuring kann auch in Funktionsparametern genutzt werden um aus übergebenen
Objekten bestimmte Attribute variablen zuzuweisen, wie in
\cref{lst:es6_destructuring_von_funktionsparametern} demonstriert wird.

\lstinputlisting[%
    caption={Destructuring von Funktionsparametern},%
    label={lst:es6_destructuring_von_funktionsparametern},%
    language=es6,%
    numbers=none,%
    float
]{destructuring_fkt.js}

\subsection{Objektorientierung}
\label{ssec:ge_objektorientierung}

Zusätzlich zu den im letzten Abschnitt aufgeführten Änderungen, wurde mit der
sechsten Edition von ECMAScript das Paradigma der Objektorientierung in das
Sprachdesign eingebaut.  So können nun mit dem Schlüsselwort \code{class}
Klassen definiert, mit \code{extends} Vererbung genutzt und mit \code{super} auf
die Oberklasse zugegriffen werden wie in \cref{lst:es6_oop} demonstriert wird.

\lstinputlisting[%
    caption={Objektorientierung in ES6},%
    label={lst:es6_oop},%
    language=es6,%
    float
]{oop.js}

\subsection{Promises}
\label{ssec:ge_promises}

Das in ES 6 eingeführte Konzept von Promises~\cite{promises} vereinfacht den
Umgang mit asynchronen Aktionen wie beispielsweise AJAX-Requests.  Promises
sind \enquote{Versprechen}, die zeitversetzt erfüllt oder abgewiesen werden
können.  Ein Promise-Objekt hat die Methoden \code{then(...)} und
\code{catch(...)} die jeweils eine Funktion übergeben bekommen.  Die der Methode
\code{then(...)} übergebene Funktion wird mit beim erfolgreichen erfüllen des
Promises aufgerufen.  Im Falle einer Exception oder dem expliziten Abweisen
des Promises wird die Funktion aufgerufen, die \code{then(...)} übergeben wurde.

Promises können verkettet werden, indem aus einer \code{then(...)} übergebenen
Funktion wieder ein Promise als Rückgabewert ausgegeben wird.  In
\cref{lst:es6_promises} wird der Umgang mit Promises anhand eines Beispiels
erläutert.

\lstinputlisting[%
	float,
	language=ES6,
	label={lst:es6_promises},
	caption={Promises}
]{promises.js}

