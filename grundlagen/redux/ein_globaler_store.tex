\subsection{Ein Globaler Store}
\label{ssec:gr_ein_globaler_store}

Im Gegensatz zu Flux, wo mehrere Stores definiert werden, befindet sich in einer
Redux-App der gesamte State in einem globalen Store.  Dies hat einige Vorteile
wie in diesem Abschnitt gezeigt wird.

Durch einen einzigen Store kann der State einer Applikation ohne zusätzlichen
Code persistiert und wiederhergestellt werden.  Dies ist vor allem bei
Integrationstest hilfreich, um bestimmte Ausgangssituationen zu simulieren.
Redux stellt zusätzlich Entwickler-Tools~\cite{reduxDevTools} zur Verfügung, die
den Store überwachen, Actions anzeigen und die daraus folgenden Veränderungen
am Store hervorheben.  Die Entwickler-Tools speichern zu jeder Action den
letzten State, sodass es möglich ist zu jedem Zeitpunkt in der History der
State-Änderungen zurückzukehren.
