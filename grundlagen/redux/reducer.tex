\subsection{Reducer}
\label{ssec:gr_reducer}

Um die Komplexität des Globalen Stores herunterzubrechen, werden sogenannte
\stichwort[Redux]{Reducer} eingeführt.  Reducer sind einfache Funktionen, die
als Parameter den aktuellen State und eine Action übergeben bekommen.  Der
Rückgabewert eines Reducers gibt den aktualisierten State an.  Durch diese
Eigenschaften sind Reducer sehr einfach zu Testen, wie in
\cref{sec:tf_reducer_tests} gezeigt wird.

Damit ein Reducer nicht den gesamten State der App kontrollieren muss, werden
mehrere Reducer durch Komposition zu einem zusammengeschlossen.  Jeder Reducer
ist also nur für einen Teil des States verantwortlich.  Die Komposition kann
beliebig tief verschachtelt werden, sodass auch komplexe States von einfachen
Reducern verwaltet werden können.  Dies wird mit der von Redux bereitgestellten
Funktion \code{combineReducers} erreicht.  Die Funktion teilt den Store unter
den übergebenen einzelnen Reducern auf und gibt einen übergeordneten Reducer als
Rückgabewert aus.

Um Reducer frei von Seiteneffekten zu halten, sind In-Place-Änderungen des
States untersagt.  Reducer müssen also auf einer Kopie des übergebenen aktuellen
States arbeiten.
