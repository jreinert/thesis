\subsection{React Components und JSX}
\label{ssec:gf_react_components_und_jsx}

React-Components bilden den View-Layer einer Flux Applikation.  Statt React
können auch beliebige andere Template-Systeme verwendet werden.  React bietet
jedoch einige Vorteile, die kurz erläutert werden sollen.

React arbeitet mit einem \stichwort[React]{Virtual-DOM}-Modell.  Das bedeutet,
dass sobald sich der State einer Component ändert, die Änderung am Virtual-DOM
vorgenommen wird.  Anschließend wird eine Differenz zum letzten Virtual-DOM
gebildet, die die minimale Anzahl an Operationen liefert um das eigentliche DOM
zu Aktualisieren.  Browser-Events werden an das Virtual-DOM gesendet und können
darüber in React-Components abgefangen werden~\cite{ReactDemystified}.
In \cref{fig:react_virtual_dom} ist der Aufbau veranschaulicht.

\begin{figure}
    \centering
    \includestandalone{fig_virtual_dom}
    \caption{React Virtual DOM}
    \label{fig:react_virtual_dom}
\end{figure}

React-Components sind Objekte und besitzen eine \code{render()}-Funktion.  Diese
Funktion wird beim Initialisieren einer Component sowie bei jeder nachfolgenden
Änderung des States aufgerufen.  Das Virtual-DOM wird Anhand des Rückgabewertes
von \code{render()} aktualisiert.  React-Components können durch Vererbung und
Mixins erweitert werden und können mit der in
\cref{ssec:ge_objektorientierung} vorgestellten Syntax definiert werden.

Um das Erzeugen von Components zu erleichtern wurde von React eine
JavaScript-Erweiterung \stichwort[React]{JSX}~\cite{JSX} eingeführt.  Durch die
Erweiterung können in JavaScript HTML-Elemente oder React-Components durch
eine XML-Artige Syntax direkt beschreiben und verwendet werden.

In \cref{lst:react_component} werden zwei beispielhafte React-Components mit JSX
definiert.  Durch eine öffnende spitze Klammer wird der JSX-Kontext eingeleitet
und durch das Schließen des Elements analog zu XML wieder beendet.  Als Elemente
können alle HTML-Elemente, sowie alle im Scope bekannten React-Components
angegeben werden.

\lstinputlisting[%
    caption={Eine React-Component in JSX},%
    label={lst:react_component},%
    language=es6,%
    float
]{react_component.js}

Innerhalb des JSX-Kontexts kann dieser ähnlich zur in \cref{ssec:ge_syntax}
beschriebenen String-Interpolation mit einer öffnenden geschweiften Klammer
unterbrochen werden.  Der Ausdruck innerhalb der geschweiften Klammern wird
ausgewertet und mit dem Ergebnis der Auswertung ersetzt.

Ab Zeile 22 in \cref{lst:react_component} wird von sogenannten
\stichwort[React]{PropTypes} Gebrauch gemacht.  PropTypes legen fest, welche
Attribute -- oder Properties -- der Component übergeben werden müssen.
Zusätzlich können die Typen der übergebenen Attribute eingeschränkt werden.
Diese Einschränkungen sind sehr hilfreich zum Aufspüren von Fehlern die sich
sonst von Component zu Kind-Component fortpflanzen würden.
