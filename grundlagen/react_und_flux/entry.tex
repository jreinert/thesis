\section{React und Flux}
\label{sec:g_react_und_flux}

In \cref{ssec:el_react_und_flux_mit_meteor} wurde bereits ein Einblick in React
und Flux gegeben, soll hier jedoch genauer betrachtet werden, da die gesamte
Frontendentwicklung auf dieser Architektur aufbaut.  Sowohl Flux~\cite{flux} als
auch React~\cite{react} sind von Facebook Inc. entwickelt worden und werden
unter der freien BSD Lizenz~\cite{bsd} vertrieben.

Zu einer Flux Architektur gehören vier Hauptbestandteile:
\emph{Actions}\index{Flux!Action}, \emph{Dispatcher}\index{Flux!Dispatcher},
\emph{Stores}\index{Flux!Store} und \emph{Views}\index{Flux!View}.  Wie in
\cref{fig:flux_data_flow} zu erkennen, führt der Datenfluss in einer Flux
Applikation ausgehend von einer Action über den Dispatcher zu einem Store und
letztendlich zu einer View.  Views können durch Aktionen von Nutzern weitere
Actions erzeugen.  Es ist nicht erlaubt innerhalb eines laufenden
Dispatch-Prozesses weitere Actions durch den Dispatcher zu versenden.  Wird dies
dennoch getan, wirft der Dispatcher eine Exception.  Somit sind Dispatch-Zyklen
ausgeschlossen.

\begin{figure}
    \centering
    \includestandalone{fig_data_flow}
    \caption{Datenfluss einer Flux Applikation}
    \label{fig:flux_data_flow}
\end{figure}

In den folgenden Abschnitten werden die einzelnen Komponenten von Flux
erläutert.

\subsection{Action}
\label{ssec:gf_action}

Eine Action ist ein einfaches JavaScript-Objekt, das ein Attribut namens
\code{type} enthält.  Das \code{type}-Attribut wird in den Stores zum
eindeutigen Feststellen der Art der Action genutzt.  Beispiele für Werte von
\code{type}:  \code{'UPDATE\_ITEM'}, \code{'FETCH\_ITEMS'}.

Üblicherweise enthalten Actions noch weitere Attribute.  Im Rahmen dieser Thesis
wurde die sogenannte
\stichwort[Flux]{Flux-Standard-Action}-Spezifikation~\cite{fsa}, kurz FSA,
eingehalten.  Diese Spezifikation sieht drei weitere Attribute für Actions vor:
\code{payload}, \code{error} und \code{meta}, wobei nur das Definieren von
\code{type} verpflichtend ist.  Eine Action darf laut FSE keine weiteren
Attribute besitzen.

Das \code{payload}-Attribut wird genutzt, um die für die jeweilige Action
relevanten Daten zu übertragen.  So könnte es beispielsweise für die Action
\code{'UPDATE\_ITEM'} als ein Objekt mit den veränderten Attributen definiert
werden.

Das Attribut \code{meta} kann zusätzliche Informationen enthalten, die nicht in
\code{payload} aufgenommen werden sollen.  Wie in \cref{sec:f_redux_middleware}
gezeigt wird, kann mithilfe von \code{meta} dem Dispatcher signalisiert werden,
dass die Action anders behandelt werden soll.

Letztlich dient das \code{error} Attribut dazu eine Action als fehlgeschlagen zu
markieren.  In diesem Fall wird von FSA empfohlen in \code{payload} ein
Error-Objekt zu übergeben.

\subsection{Store}
\label{ssec:fc-store}

\subsection{Dispatcher}
\label{ssec:gf-dispatcher}

\subsection{Action Creators}
\label{ssec:gf-action-creators}

\subsection{React Components und JSX}
\label{ssec:gf_react_components_und_jsx}

React-Components bilden den View-Layer einer Flux Applikation.  Statt React
können auch beliebige andere Template-Systeme verwendet werden.  React bietet
jedoch einige Vorteile, die kurz erläutert werden sollen.

React arbeitet mit einem \stichwort[React]{Virtual-DOM}-Modell.  Das bedeutet,
dass sobald sich der State einer Component ändert, die Änderung am Virtual-DOM
vorgenommen wird.  Anschließend wird eine Differenz zum letzten Virtual-DOM
gebildet, die die minimale Anzahl an Operationen liefert um das eigentliche DOM
zu Aktualisieren.  Browser-Events werden an das Virtual-DOM gesendet und können
darüber in React-Components abgefangen werden~\cite{ReactDemystified}.
In \cref{fig:react_virtual_dom} ist der Aufbau veranschaulicht.

\begin{figure}
    \centering
    \includestandalone{fig_virtual_dom}
    \caption{React Virtual DOM}
    \label{fig:react_virtual_dom}
\end{figure}

React-Components sind Objekte und besitzen eine \code{render()}-Funktion.  Diese
Funktion wird beim Initialisieren einer Component sowie bei jeder nachfolgenden
Änderung des States aufgerufen.  Das Virtual-DOM wird Anhand des Rückgabewertes
von \code{render()} aktualisiert.  React-Components können durch Vererbung und
Mixins erweitert werden und können mit der in
\cref{ssec:ge_objektorientierung} vorgestellten Syntax definiert werden.

Um das Erzeugen von Components zu erleichtern wurde von React eine
JavaScript-Erweiterung \stichwort[React]{JSX}~\cite{JSX} eingeführt.  Durch die
Erweiterung können in JavaScript HTML-Elemente oder React-Components durch
eine XML-Artige Syntax direkt beschreiben und verwendet werden.

In \cref{lst:react_component} werden zwei beispielhafte React-Components mit JSX
definiert.  Durch eine öffnende spitze Klammer wird der JSX-Kontext eingeleitet
und durch das Schließen des Elements analog zu XML wieder beendet.  Als Elemente
können alle HTML-Elemente, sowie alle im Scope bekannten React-Components
angegeben werden.

\lstinputlisting[%
    caption={Eine React-Component in JSX},%
    label={lst:react_component},%
    language=es6,%
    float
]{react_component.js}

Innerhalb des JSX-Kontexts kann dieser ähnlich zur in \cref{ssec:ge_syntax}
beschriebenen String-Interpolation mit einer öffnenden geschweiften Klammer
unterbrochen werden.  Der Ausdruck innerhalb der geschweiften Klammern wird
ausgewertet und mit dem Ergebnis der Auswertung ersetzt.

Ab Zeile 22 in \cref{lst:react_component} wird von sogenannten
\stichwort[React]{PropTypes} Gebrauch gemacht.  PropTypes legen fest, welche
Attribute -- oder Properties -- der Component übergeben werden müssen.
Zusätzlich können die Typen der übergebenen Attribute eingeschränkt werden.
Diese Einschränkungen sind sehr hilfreich zum Aufspüren von Fehlern die sich
sonst von Component zu Kind-Component fortpflanzen würden.

