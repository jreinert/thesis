\subsection{Action}
\label{ssec:gf_action}

Eine Action ist ein einfaches JavaScript-Objekt, das ein Attribut namens
\code{type} enthält.  Das \code{type}-Attribut wird in den Stores zum
eindeutigen Feststellen der Art der Action genutzt.  Beispiele für Werte von
\code{type}:  \code{'UPDATE\_ITEM'}, \code{'FETCH\_ITEMS'}.

Üblicherweise enthalten Actions noch weitere Attribute.  Im Rahmen dieser Thesis
wurde die sogenannte
\stichwort[Flux]{Flux-Standard-Action}-Spezifikation~\cite{fsa} (FSA)
eingehalten.  Diese Spezifikation sieht drei weitere Attribute für Actions vor:
\code{payload}, \code{error} und \code{meta}, wobei nur das Definieren von
\code{type} verpflichtend ist.  Eine Action darf laut FSE keine weiteren
Attribute besitzen.

Das \code{payload}-Attribut wird genutzt, um die für die jeweilige Action
relevanten Daten zu übertragen.  So könnte es beispielsweise für die Action
\code{'UPDATE\_ITEM'} als ein Objekt mit den veränderten Attributen definiert
werden.

Das Attribut \code{meta} kann zusätzliche Informationen enthalten, die nicht in
\code{payload} aufgenommen werden sollen.  Wie in \cref{sec:f_redux_middleware}
gezeigt wird, kann mithilfe von \code{meta} dem Dispatcher signalisiert werden,
dass die Action anders behandelt werden soll.

Letztlich dient das \code{error} Attribut dazu eine Action als fehlgeschlagen zu
markieren.  In diesem Fall wird von FSA empfohlen in \code{payload} ein
Error-Objekt zu übergeben.
