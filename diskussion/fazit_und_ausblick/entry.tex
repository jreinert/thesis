\section{Fazit und Ausblick}
\label{sec:d_fazit_und_ausblick}

Im Nachhinein betrachtet war die Zielsetzung angesichts der verfügbaren
Bearbeitungszeit zu optimistisch.  Durch den Einsatz von vielen vorher
unbekannten Technologien ist der Zeitaufwand nur schwer einschätzbar gewesen.
Der Gewinn an neuer Erkenntnis und Erfahrung war deshalb jedoch umso höher.

Das Teilprojekt Crouter hat bereits Zuspruch in der Crystal-Community gefunden
und wird aktiv weiterentwickelt.  Dasselbe gilt für die JSONApi.  Auf Basis
dieser Projekte und den aus der CCCApi erlangten Erfahrungen können weitere
JSON-basierte Backend-Implementierungen in Crystal umgesetzt werden.  Aus
CCCApi können Teile wie die Repository oder die Authentifizierungs-Middleware
und -Service-Objekte extrahiert werden.  Dies bedarf aber weiterer
Modularisierung, die sich besonders für die Repository nicht trivial gestalten
würde.  Ursprünglich geplant war eine datenbankagnostische
Repository-Bibliothek, die sich aber in der Umsetzung als zu komplex erwiesen
hat.

Um CCC im Produktiveinsatz zu nutzen sind sowohl im Frontend als auch im Backend
mehrere Weiterentwicklungen notwendig.  Eine Chat-Funktion zwischen Clients ist
beispielsweise im erreichten Zustand der Implementierung nicht eingebaut.  Im
Moment wird beim Bearbeiten von Items auch von nur einer Applikation
ausgegangen.  In Wirklichkeit müsste das System jedoch Multi-Tenancy
und mehrere Applikationen pro Tenant unterstützen.  Dies erhöht die Komplexität
der Authentifizierung und Rechteverwaltung.  Der \enquote{Grundstein} für die
Umsetzung ist jedoch gelegt und die Recherche und Abwägung der Technologien
getan.

Im großen Ganzen konnte im Verlauf der Arbeit vieles gelernt werden.  Die
Zielsetzung war anspruchsvoll und fordernd, die Umsetzung Abwechslungsreich.
Es boten sich viele interessante Herausforderungen sowohl in der Planung als
auch in der eigentlichen Entwicklung. Die Arbeit hat trotz der großen
Umstellung nach der Entscheidung Meteor auszuschließen ein zufriedenstellendes
Ergebnis hervorgebracht.  Die Implementierung konnte so weit durchgeführt
werden, dass das Erfüllen aller gesetzten Anforderungen demonstriert werden
kann.
