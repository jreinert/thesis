\section{Fazit und Ausblick}
\label{sec:d_fazit_und_ausblick}

Das Teilprojekt Crouter hat bereits Zuspruch in der Crystal-Community gefunden
und wird aktiv weiterentwickelt.  Dasselbe gilt für die JSONApi.  Auf Basis
dieser Projekte und den aus der CCCApi erlangten Erfahrungen können weitere
JSON-basierte Backend-Implementierungen in Crystal umgesetzt werden.  Aus
CCCApi können Teile wie die Repository oder die Authentifizierungs-Middleware
und -Service-Objekte extrahiert werden.  Das erfordert aber weitere
Modularisierung, die sich besonders für die Repository nicht trivial gestalten
würde.  Ursprünglich geplant war eine datenbankagnostische
Repository-Bibliothek, die sich aber im nachhinein in der Umsetzung als zu
komplex erwiesen hat.

Das System selbst bietet eine Grundlage, die durch Weiterentwicklungen im
Bereich der Client-zu-Client-Kommunikation, beispielsweise durch eine
Chat-Funktion, den Kollaborationsaspekt des Systems ohne größeren
Entwicklungsaufwand stärker präsentieren kann.  Um das System mehr als nur
einem Kunden anbieten zu können, besteht die Möglichkeit Multi-Tenancy mit
mehreren Applikationen (Mengen von Items) pro Tenant nachzurüsten.  Dies erhöht
die Komplexität der Authentifizierung und Rechteverwaltung.

Durch den Einsatz von vielen vorher unbekannten Technologien war der
Zeitaufwand nur schwer einzuschätzen.  Der Gewinn an neuer Erfahrung war
dadurch jedoch umso höher.  Die Umsetzung mit Meteor führte zur Erkenntnis,
dass durch eine Eigenimplementierung wesentliche Komponenten optimiert werden
konnten.  Die Entscheidung, Meteor zu ersetzen, führte zwar zu einer großen
Umstellung, erbrachte jedoch letztendlich einen erheblichen Mehrwert für das
Endresultat.

Es wurde eine umfangreiche Analyse der aktuell verfügbaren Web-Technologien für
ein kollaborativ nutzbares CMS durchgeführt und anhand der Recherche ein System
implementiert, das alle gegebenen Anforderungen erfüllt.  Die Thesis bot eine
optimale Möglichkeit zur Ausarbeitung einer anspruchsvollen Zielsetzung sowie
einer abwechslungsreichen Implementierung und stellte insgesamt eine sehr
lehrreiche Erfahrung dar.
