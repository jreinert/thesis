\section{Fazit und Ausblick}
\label{sec:d_fazit_und_ausblick}

Das Teilprojekt Crouter hat bereits Zuspruch in der Crystal-Community gefunden
und wird aktiv weiterentwickelt.  Dasselbe gilt für die JSONApi.  Auf Basis
dieser Projekte und den aus der CCCApi erlangten Erfahrungen können weitere
JSON-basierte Backend-Implementierungen in Crystal umgesetzt werden.  Aus
CCCApi können Teile wie die Repository oder die Authentifizierungs-Middleware
und -Service-Objekte extrahiert werden.  Dies bedarf aber weiterer
Modularisierung, die sich besonders für die Repository nicht trivial gestalten
würde.  Ursprünglich geplant war eine datenbankagnostische
Repository-Bibliothek, die sich aber in der Umsetzung als zu komplex erwiesen
hat.

Das System selbst kann ebenfalls durch mehrere Weiterentwicklungen profitieren.
Obwohl die Grundlagen der Client-zu-Client-Kommunikation implementiert wurden,
kann in diesem Bereich beispielsweise durch eine Chat-Funktion der
Kollaborationsaspekt des Systems ohne größerem Entwicklungsaufwand
unterstrichen werden.  Um das System mehreren Kunden anbieten zu können kann
Multi-Tenancy mit mehreren Applikationen (Mengen von Items) pro Tenant
nachgerüstet werden.  Dies erhöht die Komplexität der Authentifizierung und
Rechteverwaltung.

Durch den Einsatz von vielen vorher unbekannten Technologien ist der
Zeitaufwand nur schwer einschätzbar gewesen.  Der Gewinn an neuer Erkenntnis
und Erfahrung war dadurch jedoch umso höher.  Im großen Ganzen wurde im Verlauf
der Arbeit vieles gelernt,  die Zielsetzung war anspruchsvoll und fordernd, die
Umsetzung Abwechslungsreich.  Es boten sich viele interessante
Herausforderungen sowohl in der Planung als auch in der eigentlichen
Entwicklung.  Die Arbeit hat trotz der großen Umstellung nach der Entscheidung
Meteor auszuschließen das gewünschte Ergebnis hervorgebracht:  Es wurde eine
umfangreiche Analyse der aktuell verfügbaren Web-Technologien für ein
kollaborativ nutzbares CMS durchgeführt und anhand der Recherche eine Umsetzung
gefunden, die alle gegebenen Anforderungen erfüllt.
